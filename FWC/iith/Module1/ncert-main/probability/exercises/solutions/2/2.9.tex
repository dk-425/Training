Let the input variables $X \in \{0,1\}$ and $Y \in \{0,1\}$ be defined according to the table \ref{tab:table1} .

\begin{table}[!ht]
\centering
\resizebox{\columnwidth}{!}{\begin{tabular}{|c|c|c|} 
\hline
Input Variable & Value & Description \\
\hline
\multirow{2}{*}{X} & 0 & Absence of disease in test \\ \cline{2-3} 
                   & 1  & Presence of disease in test \\ \hline
\multirow{2}{*}{Y} & 0 & Absence of disease in reality \\ \cline{2-3} 
                   & 1 & Presence of disease in reality \\ \hline
\end{tabular}}
\caption{Input Variables}
\label{tab:table1}
\end{table}
Given data of the question is presented in the table \ref{tab:table2} .

\begin{table}[!ht]
\centering
\begin{tabular}{|c|c|c|} 
\hline
S.No. & Expression & Value \\
\hline 
1. &  Pr$(X=1|Y=1)$ & $\frac{99}{100}$ \\ 
\hline 
2. & Pr$(X=1|Y=0)$ & $\frac{0.5}{100} = \frac{1}{200}$ \\ 
\hline 
3. & Pr(Y=1) & $\frac{0.1}{100} = \frac{1}{1000}$ \\ 
\hline 
4. & Pr(Y=0) & $\frac{99.9}{100} = \frac{999}{1000}$ \\ 
\hline 
\end{tabular}
\caption{Given Data}
\label{tab:table2}
\end{table}
Hence,probability that a person has the disease given that his test result is positive ,is given by
\begin{align}
    \text{Pr}(Y=1|X=1) &= \frac{\text{Pr}(X=1|Y=1)\text{Pr}(Y=1)}{ \sum_{i=0}^{1}\text{Pr}(X=1|Y=i)\text{Pr}(Y=i)}
    \\
    &= \frac{\frac{99}{100} \times \frac{1}{1000}}{\frac{1}{200} \times \frac{999}{1000} + \frac{99}{100} \times \frac{1}{1000}}
    \\
    &= \frac{22}{133}
\end{align}