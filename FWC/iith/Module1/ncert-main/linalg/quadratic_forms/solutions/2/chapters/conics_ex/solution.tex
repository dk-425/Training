\begin{enumerate}


\item To solve the equation - $3x^2-4x+\frac{20}{3} = 0$ 
\begin{flushleft}
The given equation can be represented as follows in the vector form
\end{flushleft}
\begin{align}
\vec{x}^T 
\myvec{
3 & 0 \\
0 & 0
}
\vec{x} + 
\myvec{
-4 & 0 
}
\vec{x} + \frac{20}{3} = 0
\end{align}

\begin{align}
\vec{x} = \myvec{x\\0} \\
3x^2-4x+\frac{20}{3} &= 0 \\
\brak{x-\myvec{\frac{2}{3}\\ \frac{2\sqrt{14}}{3}}}\brak{x-\myvec{\frac{2}{3}\\\frac{-2\sqrt{14}}{3}}} &= 0 \\
x = \myvec{\frac{2}{3}\\ \frac{2\sqrt{14}}{3}},\myvec{\frac{2}{3}\\ \frac{-2\sqrt{14}}{3}}
\end{align}

Figure \ref{fig:5.2.2_quadeq1_conics_ex} show that the equation does not intersect the x-axis hence there are no real roots. 

\begin{figure}[!ht]
\centering
\includegraphics[width=\columnwidth]{./solutions/2/figs/conics_ex/quadratic_equation_1.eps}
\caption{$3x^2-4x+\frac{20}{3}$ generated using python}
\label{fig:5.2.2_quadeq1_conics_ex}
\end{figure} 
%%%%%%%%%%%%%%%%%%%%%%
\item To solve the equation - $x^2-2x+\frac{3}{2} = 0$ 
\begin{flushleft}
The given equation can be represented as follows in the vector form
\end{flushleft}
\begin{align}
\vec{x}^T 
\myvec{
1 & 0 \\
0 & 0
}
\vec{x} + 
\myvec{
-2 & 0 
}
\vec{x} + \frac{3}{2} = 0
\end{align}

\begin{align}
\vec{x} = \myvec{x\\0}\\
x^2-2x+\frac{3}{2} &= 0 \\
\brak{x-\myvec{1 \\ \sqrt{2}}}\brak{x - \myvec{1\\ -\sqrt{2}}} &= 0 \\
x = \myvec{1\\ \sqrt{2}}, \myvec{1\\ -\sqrt{2}}
\end{align}

Figure \ref{fig:5.2.2_quadeq2_conics_ex} show that the equation does not intersect the x-axis hence there are no real roots.

\begin{figure}[!ht]
\centering
\includegraphics[width=\columnwidth]{./solutions/2/figs/conics_ex/quadratic_equation_2.eps}
\caption{$x^2-2x+\frac{3}{2}$ generated using python}
\label{fig:5.2.2_quadeq2_conics_ex}
\end{figure} 
%%%%%%%%%%%%%%%%%%%%%%
\item To solve the equation - $27x^2-10x+1 = 0$ 
\begin{flushleft}
The given equation can be represented as follows in the vector form
\end{flushleft}
\begin{align}
\vec{x}^T 
\myvec{
27 & 0 \\
0 & 0
}
\vec{x} + 
\myvec{
-10 & 0 
}
\vec{x} + 1 = 0
\end{align}

\begin{align}
\vec{x} = \myvec{x\\0} \\
27x^2-10x+1 &= 0 \\
\brak{x-\myvec{\frac{5}{27} \\ \  \frac{\sqrt{2}}{27}}}\brak{x - \myvec{\frac{5}{27} \\ \ \frac{-\sqrt{2}}{27}}} &= 0 \\
x = \myvec{\frac{5}{27} \\ \  \frac{\sqrt{2}}{27}}, \myvec{\frac{5}{27} \\ \  \frac{-\sqrt{2}}{27}}
\end{align}

Figure \ref{fig:5.2.2_quadeq3_conics_ex} show that the equation does not intersect the x-axis hence there are no real roots.

\begin{figure}[!ht]
\centering
\includegraphics[width=\columnwidth]{./solutions/2/figs/conics_ex/quadratic_equation_3.eps}
\caption{$27x^2-10x+1$ generated using python}
\label{fig:5.2.2_quadeq3_conics_ex}
\end{figure} 
%%%%%%%%%%%%%%%%%%%%%%
\item To solve the equation - $21x^2-28x+10 = 0$ 
\begin{flushleft}
The given equation can be represented as follows in the vector form
\end{flushleft}
\begin{align}
\vec{x}^T 
\myvec{
21 & 0 \\
0 & 0
}
\vec{x} + 
\myvec{
-28 & 0 
}
\vec{x} + 10 = 0
\end{align}

\begin{align}
\vec{x} = \myvec{x\\0}\\
21x^2-28x+10 &= 0 \\
\brak{x-\myvec{\frac{2}{3} \\\  \frac{\sqrt{14}}{21}}}\brak{x - \myvec{\frac{2}{3} \\\  \frac{-\sqrt{14}}{21}}} &= 0 \\
x = \myvec{\frac{2}{3} \\\  \frac{\sqrt{14}}{21}}, \myvec{\frac{2}{3} \\\  \frac{-\sqrt{14}}{21}}
\end{align}

Figure \ref{fig:5.2.2_quadeq4_conics_ex} show that the equation does not intersect the x-axis hence there are no real roots.

\begin{figure}[!ht]
\centering
\includegraphics[width=\columnwidth]{./solutions/2/figs/conics_ex/quadratic_equation_4.eps}
\caption{$21x^2-28x+10$ generated using python}
\label{fig:5.2.2_quadeq4_conics_ex}
\end{figure} 
%%%%%%%%%%%%%%%%%%%%%%

The  following Python code generates Fig.\ref{fig:5.2.2_quadeq1_conics_ex}, \ref{fig:5.2.2_quadeq2_conics_ex}, \ref{fig:5.2.2_quadeq3_conics_ex} and \ref{fig:5.2.2_quadeq4_conics_ex} 

\begin{lstlisting}
solutions/2/codes/conics_ex/conics_ex.py
\end{lstlisting}
\end{enumerate}
