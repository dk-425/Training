
\begin{figure}[!ht]
\centering
\resizebox{\columnwidth}{!}{\begin{tikzpicture}[scale=1.5,>=stealth,point/.style={draw,circle,fill = black,inner sep=0.5pt},]
      
%Labeling points
\node (A) at (0, 0)[point,label=below left:$A$] {};
\node (C) at (5, 0)[point,label=below right:$C$] {};
\node (B) at (7, 5.9)[point,label=above right:$B$] {};
\node (D) at (0.86, 4.92)[point,label=above left:$D$] {};
\node (O) at (1.95, 1.64)[point,label=above left:$O$] {};
\node (P1) at (1.95, 0)[point,label=below left:$P_1$] {};
\node (P2) at (0.33, 1.92)[point,label=above left:$P_2$] {};

%Drawing quad ABCD
\draw (A) -- node[below=5pt]{$\textrm{a}$}(C) -- (B) -- (D) -- (A);
\draw[dotted] (A) -- (B)(O) -- (P1) -- (P2) -- (O);

%marking angles
\tkzMarkRightAngle[fill=blue!20, mark=|](C,P1,O)
\tkzMarkRightAngle[fill=blue!20, mark=|](O,P2,D)
\tkzMarkAngle[fill=green!20, mark=|](B,A,D)
\tkzMarkAngle[fill=green!20, mark=|](C,A,B)

%marking lines
\tkzMarkSegment[color=black,pos=0.5,mark=s||](A,D)
\tkzMarkSegment[color=black,pos=0.5,mark=s||](A,C)

\end{tikzpicture}}
\caption{}
\label{fig:8.1.23_quad}	
\end{figure}
 
\item {\em Construxtion: }See  Fig. \ref{fig:8.1.23_quad}.  The input parameters are
%
\begin{align}
\vec{A} &=\myvec{0\\0} \label{eq:8.1.23_constr_a}\\
\vec{C} &= \myvec{a\\0}, \label{eq:8.1.23_constr_c}\\ 
\vec{D} &= a\myvec{\cos{\theta}\\\sin{\theta}}\label{eq:8.1.23_constr_b}
\end{align}
\subitem Let 
%
\begin{align}
\begin{split}
\vec{P}_1 &= \frac{\vec{C}}{\norm{ \vec{C} }} \\
\vec{P}_2 &= \frac{\vec{D}}{\norm{ \vec{D}}} 
\end{split}
 \label{eq:8.1.23_p1p2} 
\end{align}
If $AB$ be the angle bisector, it is easy to show that 
\begin{align}
P_1P_2 \perp  AB \\
\implies \brak{\vec{P}_1 - \vec{P}_2}^T \brak{\vec{B}-\vec{A}}
 \label{eq:8.1.23_Proofeq1} 
\end{align}
However, from  \eqref{eq:8.1.23_p1p2} ,
\begin{align}
\norm{ \vec{P}_1} &= \norm{ \vec{P}_2}  = 1\\
\implies \norm{ \vec{P}_1}^2 &= \norm{ \vec{P}_2}^2 
\\
\text{or, }\brak{\vec{P}_1-\vec{P}_2}^T\brak{\vec{P}_1+\vec{P}_2} &= 0 \\
\implies \vec{P}_1 + \vec{P}_2 \perp  \vec{P}_1-\vec{P}_2 \label{eq:8.1.23_Proofeq2}
\end{align}
From \eqref{eq:8.1.23_Proofeq1} and \eqref{eq:8.1.23_Proofeq2}
\begin{align}
\vec{B} &= \lambda\brak{\vec{P}_1+\vec{P}_2} 
\\
\implies \vec{B} &=  \lambda\brak{\frac{\vec{C}}{\norm{ \vec{C}}}+\frac{\vec{D}}{\norm{ \vec{D}}}}
\end{align}
when $\vec{A}$ is at the origin.  In general, 
\begin{align}
\vec{B} =  \vec{A} + \lambda\brak{\frac{\vec{C}-\vec{A}}{\norm{ \vec{C}-\vec{A}}}+\frac{\vec{D}-\vec{A}}{\norm{ \vec{D}-\vec{A}}}}
\end{align}
\item Using SAS congruence, it is trivial to show that $BC = BD$
