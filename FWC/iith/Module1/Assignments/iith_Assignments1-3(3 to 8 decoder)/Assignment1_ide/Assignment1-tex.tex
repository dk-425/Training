
\documentclass[journal,12pt,twocolumn]{IEEEtran}
\title{
3 to 8 Decoder through Arduino UNO
}
\author{Kukunuri Sampath Govardhan}
\begin{document}
\maketitle
\tableofcontents
\begin{abstract}
This document shows how to use Arduino UNO as a 3 to 8 Decoder
\end{abstract}
\section{Components}
\begin{table}[h]
    \centering
    \begin{tabular}{| c | c | c |}
       \hline
       \textbf{Component}  &  \textbf{Value}  &  \textbf{Qunatity}\\
       \hline
         Resistor  &  220Ohm  &  8  \\
         \hline
         LED  &  Red  &  8  \\
         \hline
         Arduino  & UNO & 1  \\
         \hline
         Jumper Wires  &  M-M  &  20  \\
         \hline
         BreadBoard  &    &  1\\
         \hline
         
    \end{tabular}
    \caption{}
    \label{tab:my_label}
\end{table}
\section{Hardware}
\textbf{Problem 2.1} Make connections between the Arduino UNO and LED's as shown in Table 2 \\

\textbf{Problem 2.2} Connect anodes of LED's to the pins using resistors and cathodes to ground(gnd).\\
\begin{table}[h]
    \centering
    \begin{tabular}{| c | c | c | c | c | c | c | c |}
        \hline
         d2 & d3 & d4 & d5 & d6 & d7 & d8 & d9   \\
       \hline
       led1 & led2 & led3 & led4 & led5 & led6 & led7 & led8  \\
         \hline
    \end{tabular}
    \caption{}
    \label{tab:my_label}
\end{table}
\\
\textbf{Problem 2.3} Connect the pins X, Y, Z to arduino as shown in Table 3.\\

    \begin{table}[h]
    \centering
    \begin{tabular}{| c | c | c |}
    \hline
    \textbf{X} & \textbf{Y} & \textbf{Z} \\
    \hline
    d10 & d11 & d12  \\
    \hline
    \end{tabular}
    \caption{Truth Table}
    \label{tab:my_label}
 \end{table}


\section{Software}

\textbf{Problem 3.1} Now execute the following program and verify all the outputs as mentioned in Truth table (Table 4) by modifying the inputs X, Y, Z to 0's and 1's respectively. \\
\begin{table}[h]
    \centering
    \begin{tabular}{| c |}
    \hline
    wget https://github.com/mygit-sampath-govardhan/fwc-iith-assignments/\\blob/87ed9e342621506842030b41a700367a3d88d0a8/\\Assignment-1/Assignment1-code.cpp\\
    \hline
    \end{tabular}
\end{table}

\textbf{Note:} Output pins d2-d9 are referenced as A-H respectively and input pins d10-d12 are referred as X,Y,Z respectively.\\

    \begin{table}[h]
    \centering
    \begin{tabular}{| c | c | c || c | c | c | c | c | c | c | c |}
    \hline
    \textbf{X} & \textbf{Y} & \textbf{Z} & \textbf{A} & \textbf{B} & \textbf{C} & \textbf{D} & \textbf{E} & \textbf{F} & \textbf{G} & \textbf{H} \\
    \hline
    0 & 0 & 0 & 0 & 0 & 0 & 0 & 0 & 0 & 0 & 1  \\
    \hline
    0 & 0 & 1 & 0 & 0 & 0 & 0 & 0 & 0 & 1 & 0  \\
    \hline
    0 & 1 & 0 & 0 & 0 & 0 & 0 & 0 & 1 & 0 & 0  \\
    \hline
    0 & 1 & 1 & 0 & 0 & 0 & 0 & 1 & 0 & 0 & 0  \\
    \hline
    1 & 0 & 0 & 0 & 0 & 0 & 1 & 0 & 0 & 0 & 0  \\
    \hline
    1 & 0 & 1 & 0 & 0 & 1 & 0 & 0 & 0 & 0 & 0  \\
    \hline
    1 & 1 & 0 & 0 & 1 & 0 & 0 & 0 & 0 & 0 & 0  \\
    \hline
    1 & 1 & 1 & 1 & 0 & 0 & 0 & 0 & 0 & 0 & 0  \\
    \hline
    \end{tabular}
    \caption{Truth Table}
    \label{tab:my_label}
 \end{table}
 \\
\textbf{Solution :}  In the Truth table (Table3) X,Y,Z are inputs and A,B,C,D,E,F,G,H are outputs. This table represents the system that behaves as a 3 to 8 decoder. Using Boolean logic, \\
 \begin{center}
     H= X' Y' Z'\\
 \end{center}
  \begin{center}
      G= X' Y' Z\\
 \end{center}
  \begin{center}
      F= X' Y Z'\\
 \end{center}
  \begin{center}
      E= X' Y Z\\
 \end{center}
  \begin{center}
     D= X Y' Z'\\
 \end{center}
  \begin{center}
     C= X Y' Z\\
 \end{center}
  \begin{center}
     B= X Y Z'\\
 \end{center}
  \begin{center}
     A= X Y Z\\
 \end{center} 
 \section{Conclusion}
 A 3 to 8 decoder has 3 inputs and 8 outputs are generated using these 3 inputs.\\
 \\
  Here 3 to 8 decoder has been successfully verified.\\
\end{document}

