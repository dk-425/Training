\renewcommand{\theequation}{\theenumi}
\begin{enumerate}[label=\arabic*.,ref=\thesubsection.\theenumi]
\numberwithin{equation}{enumi}
\item 	In Fig. \ref{fig:tri_sum_angle}, the sum of all the angles on the top or bottom side of the straight line $XY$ is $180\degree$.


\begin{figure}[!ht]
	\begin{center}
		\resizebox{\columnwidth}{!}{%Code by GVV Sharma
%December 6, 2019
%released under GNU GPL
%Sum of the angles of a right angled triangle 

\begin{tikzpicture}
[scale=2,>=stealth,point/.style={draw,circle,fill = black,inner sep=0.5pt},]

%Triangle sides
\def\a{4}
\def\c{3}

%Section Ratio
\def\k{1.2}


%Labeling points
\node (A) at (0,\c)[point,label=above right:$A$] {};
\node (B) at (0, 0)[point,label=below left:$B$] {};
\node (C) at (\a, 0)[point,label=below right:$C$] {};

%Translating coordinates
\node (Y) at ($(A) + (1,0)$)[point,label=above right:$Y$] {};
\node (X) at ($(A) - (1,0)$)[point,label=above right:$X$] {};
\node (T) at ($(A) + (0,1)$)[point,label=above right:$T$] {};

%Section formula
\node (V) at ($ (C)!\k!(A) $)[point,label=above right:$V$] {};

%Drawing triangle ABC
\draw (A) -- node[left] {$\textrm{c}$} (B) -- node[below] {$\textrm{a}$} (C) -- node[above,xshift=2mm] {$\textrm{b}$} (A);

%Joining other points
\draw (Y)--(A);
\draw (X)--(A);
\draw (T)--(A);
\draw (V)--(A);

%Drawing and marking angles
\tkzMarkAngle[fill=orange!40,size=0.5cm,mark=](A,C,B)
\tkzMarkAngle[fill=orange!40,size=0.5cm,mark=](V,A,X)
\tkzMarkRightAngle[fill=blue!20,size=.3](A,B,C)
\tkzMarkRightAngle[fill=blue!20,size=.3](B,A,X)
\tkzLabelAngle[pos=0.65](A,C,B){$\theta$}
\tkzLabelAngle[pos=0.65](V,A,X){$\theta$}

\end{tikzpicture}
}
	\end{center}
	\caption{Sum of angles of a triangle}
	\label{fig:tri_sum_angle}	
\end{figure}



\item
In Fig. \ref{fig:tri_sum_angle}, the straight line making an angle of $90\degree$ to the side $AB$ is said to be parallel to the side $BC$. Note there is an angle at $A$ that is equal to $\theta$.  This is one property of parallel lines.  Thus, $\angle YAZ = 90\degree$.


\item
	Show that $\angle VAT = 90\degree - \theta$
		
	\solution Considering the line $XAT$,
	\begin{align}
	\theta + 90\degree + \angle VAT &= 180\degree \\
	\Rightarrow  \angle VAT =  90\degree - \theta
	\end{align}

\item
	\label{prob:tri_compl_angle}
	Show that $\angle BAC = 90\degree - \theta$.
	
	\solution Consider the line $VAB$ and and use the approach in the previous problem.  Note that this implies that $\angle VAT = \angle BAC$.  Such angles are known as vertically opposite angles. 
	 
\item
Sum of the angles of a triangle is equal to $180\degree$.
%
\item Draw Fig. \ref{fig:tri_sum_angle} for $a = 4, c =3$.
%
\\
\solution Problem \ref{const:tri_right_angle} is used to draw $\triangle ABC$.  The remaining points are obtained as

\begin{align}
\vec{Y} &= \vec{A} + \myvec{1\\0} = \myvec{1\\3}
\\
\vec{X} &= \vec{A} - \myvec{1\\0} = \myvec{-1\\3}
\\
\vec{T} &= \vec{A} + \myvec{0\\1} = \myvec{0\\4}
\end{align}
%
and 
\begin{align}
\frac{VC}{AC} &= k+1
\\
\label{eq:tri_section_formula}
\implies \vec{A} &= \frac{k\vec{C} + \vec{V}}{k+1}  
\\
\implies \vec{V} &= {\brak{k+1}\vec{A} - k\vec{C}}
\end{align}
%
for $k = 0.2$. \eqref{eq:tri_section_formula} is known as the {\em section formula}.
%
The python code for  Fig. \ref{fig:tri_sum_angle} is
\begin{lstlisting}
codes/triangle/tri_sum_angle.py
\end{lstlisting}
%
and the equivalent latex-tikz code is
%
\begin{lstlisting}
figs/triangle/tri_sum_angle.tex
\end{lstlisting}
\end{enumerate}
