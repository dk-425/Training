 \begin{figure}[!ht]
\centering
\resizebox{\columnwidth}{!}{\begin{tikzpicture} 
        \coordinate (B) at (2.5, -2.5) {};
        \coordinate (A) at (0, 0) {};
        \coordinate (P) at (5, 0) {};
        \coordinate (C) at (2.5, 2.5) {};

        \draw (C)node[above]{$C$}--(A)node[below]{$A$}--(P)node[above]{$P$}--cycle;
        \draw (P)node[above]{$P$}--(A)node[below]{$A$}--(B)node[below]{$B$}--cycle;
\tkzMarkRightAngle[size=.2](P,C,A);
\tkzLabelAngle[dist=.5](P,A,B){};
\tkzMarkRightAngle[size=.2](P,B,A);
\tkzLabelAngle[dist=.5](P,A,C){};
\end{tikzpicture}
}
\caption{figure}
\label{eq:solutions/1/18/fig1}
\end{figure}
\begin{enumerate}
    \item Here, the following information is given:
    \begin{align}
    \norm{\vec{P}-\vec{B}}=\norm{\vec{P}-\vec{C}} \label{eq:solutions/1/18/eq2.1}
    \end{align}
    \item The lines $PB$ is the perpendicular to line $AB$ and 
    $PC$ is the perpendicular to line $AC$:
    \begin{align}
    \myvec{\vec{P}-\vec{B}}^T\myvec{\vec{A}-\vec{B}}=0 \label{eq:solutions/1/18/eq2.2}
    \implies \cos\angle PBA=0 
    \end{align}
    \begin{align}
    \myvec{\vec{P}-\vec{C}}^T\myvec{\vec{A}-\vec{C}}=0 \label{eq:solutions/1/18/eq2.3}
    \implies \cos\angle PCA=0 
    \end{align}
\end{enumerate}
We know that 
\begin{align}
\norm{\vec{P-A}}^2=(\vec{P}-\vec{A})^T(\vec{P}-\vec{A})
\end{align}
\begin{align}
 \begin{split}
(\vec{P}-\vec{A})^T(\vec{P}-\vec{A})=(\vec{P}-\vec{B}+\vec{B}-\vec{A})^T(\vec{P}-\vec{B}+\vec{B}-\vec{A})
\end{split}
\end{align}
\begin{align}
 \begin{split}
\norm{\vec{P-A}}^2=\norm{\vec{P-B}}^2+\norm{\vec{B-A}}^2\\ 
+2\norm{\vec{A}-\vec{P}}\norm{\vec{B}-\vec{A}}\cos\angle PBA \\
\end{split}
\end{align}
using \eqref{eq:solutions/1/18/eq2.2}
\begin{align}
 \begin{split}
 \implies \norm{\vec{P-A}}^2=\norm{\vec{P-B}}^2+\norm{\vec{B-A}}^2 \label{eq:solutions/1/18/eq2.4} \\
 \end{split}
\end{align}
Similarly
\begin{align}
 \begin{split}
(\vec{P}-\vec{A})^T(\vec{P}-\vec{A})=(\vec{P}-\vec{C}+\vec{C}-\vec{A})^T(\vec{P}-\vec{C}+\vec{C}-\vec{A})
\end{split}
\end{align}
\begin{align}
 \begin{split}
\norm{\vec{P-A}}^2=\norm{\vec{P-C}}^2+\norm{\vec{C-A}}^2\\ 
+2\norm{\vec{A}-\vec{P}}\norm{\vec{C}-\vec{A}}\cos\angle PCA \\
\end{split}
\end{align}
using \eqref{eq:solutions/1/18/eq2.3}
\begin{align}
 \begin{split}
 \implies \norm{\vec{P-A}}^2=\norm{\vec{P-C}}^2+\norm{\vec{C-A}}^2 \label{eq:solutions/1/18/eq2.5}
 \end{split}
\end{align}
From \eqref{eq:solutions/1/18/eq2.4} and \eqref{eq:solutions/1/18/eq2.5} and substituting \eqref{eq:solutions/1/18/eq2.1}
\begin{align}
\norm{\vec{P-C}}^2+\norm{\vec{C-A}}^2=\norm{\vec{P-B}}^2+\norm{\vec{B-A}}^2 \\
\implies \norm{\vec{C-A}}^2=\norm{\vec{B-A}}^2
\implies \norm{\vec{C-A}}=\norm{\vec{B-A}} \label{eq:solutions/1/18/eq2.6}
\end{align}
We know that 
\begin{align}
 \cos\angle BAP=
   \frac{(\vec P -\vec A)^T(\vec{A}-\vec{B})}{\norm{\vec{P}-\vec{A}} \norm{\vec{A}-\vec{B}} } \\
   =  \frac{(\vec P -\vec{B} +\vec{B} -\vec A)^T(\vec{A}-\vec{B})}{\norm{\vec{P}-\vec{A}} \norm{\vec{A}-\vec{B}} } \\
   =\frac{(\vec P -\vec B)^T (\vec{A}-\vec{B})- \norm{\vec{B-A}}^2}{\norm{\vec{P}-\vec{A}} \norm{\vec{A}-\vec{B}} }
\end{align}
Using \eqref{eq:solutions/1/18/eq2.2}
\begin{align}
\cos\angle BAP=  \frac{\norm{\vec{B-A}}}{\norm{\vec{P}-\vec{A}}  }\label{eq:solutions/1/18/eq2.7}
\end{align}
Similarly,
\begin{align}
 \cos\angle CAP =
 \frac{(\vec P -\vec A)^T(\vec{A}-\vec{C})}{\norm{\vec{P}-\vec{A}} \norm{\vec{A}-\vec{C}} } \\
   =  \frac{(\vec P -\vec{C} +\vec{C} -\vec A)^T(\vec{A}-\vec{C})}{\norm{\vec{P}-\vec{A}} \norm{\vec{A}-\vec{C}} } \\
   =\frac{(\vec P -\vec C)^T (\vec{A}-\vec{C})- \norm{\vec{C-A}}^2}{\norm{\vec{P}-\vec{A}} \norm{\vec{A}-\vec{C}} }
\end{align}
Using \eqref{eq:solutions/1/18/eq2.3}
\begin{align}
\cos\angle CAP=  \frac{\norm{\vec{C-A}}}{\norm{\vec{P}-\vec{A}}  }\label{eq:solutions/1/18/eq2.8}
  \end{align}
Using \eqref{eq:solutions/1/18/eq2.6} in \eqref{eq:solutions/1/18/eq2.7} and \eqref{eq:solutions/1/18/eq2.8},
\begin{align}
\cos\angle BAP=\cos\angle CAP \\
\implies \angle BAP=\angle CAP
\end{align}
The line AP bisects the angle between them.


