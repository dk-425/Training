First we find orthogonal vectors $\vec{m_1}$ and $\vec{m_2}$ to the given normal vector $\vec{n}$. Let, $\vec{m}$ = $\myvec{a\\b\\c}$, then
\begin{align}
\vec{m^T}\vec{n} = 0\\
\implies\myvec{a&b&c}\myvec{2\\-1\\2} = 0\\
\implies2a-b+2c = 0\\
\intertext{Putting a=1 and b=0 we get,}
\vec{m_1} = \myvec{1\\0\\1}\\
\intertext{Putting a=0 and b=1 we get,}
\vec{m_2} = \myvec{0\\1\\1}
\end{align}
Now we solve the equation,
\begin{align}
\vec{M}\vec{x} = \vec{b}
\label{eq:solutions/3/6/eq1}
\\
\intertext{Putting values in \eqref{eq:solutions/3/6/eq1},}
\myvec{1&0\\0&1\\1&1}\vec{x} = \myvec{3\\-2\\1} \label{eq:solutions/3/6/eq2}
\end{align}
Now, to solve \eqref{eq:solutions/3/6/eq2}, we perform Singular Value Decomposition on $\vec{M}$ as follows,
\begin{align}
\vec{M}=\vec{U}\vec{S}\vec{V}^T\label{eq:solutions/3/6/eqSVD}
\end{align}
Where the columns of $\vec{V}$ are the eigen vectors of $\vec{M}^T\vec{M}$ ,the columns of $\vec{U}$ are the eigen vectors of $\vec{M}\vec{M}^T$ and $\vec{S}$ is diagonal matrix of singular value of eigenvalues of $\vec{M}^T\vec{M}$.
\begin{align}
\vec{M}^T\vec{M}=\myvec{2&1\\1&2}\label{eq:solutions/3/6/eqMTM}\\
\vec{M}\vec{M}^T=\myvec{1&0&1\\0&1&1\\1&1&2}
\end{align}
From \eqref{eq:solutions/3/6/eq1} putting \eqref{eq:solutions/3/6/eqSVD} we get,
\begin{align}
\vec{U}\vec{S}\vec{V}^T\vec{x} & = \vec{b}\\
\implies\vec{x} &= \vec{V}\vec{S_+}\vec{U^T}\vec{b}\label{eq:solutions/3/6/eqX}
\end{align}
Where $\vec{S_+}$ is Moore-Penrose Pseudo-Inverse of $\vec{S}$.Now, calculating eigen value of $\vec{M}\vec{M}^T$,
\begin{align}
\mydet{\vec{M}\vec{M}^T - \lambda\vec{I}} &= 0\\
\implies\myvec{1-\lambda&0&1\\0&1-\lambda&1\\1&1&2-\lambda} &=0\\
\implies\lambda^3-4\lambda^2+3\lambda &=0
\end{align}
Hence eigen values of $\vec{M}\vec{M}^T$ are,
\begin{align}
\lambda_1 &= 3\\
\lambda_2 &= 1\\
\lambda_3 &= 0
\end{align}
Hence the eigen vectors of $\vec{M}\vec{M}^T$ are,
\begin{align}
\vec{u}_1=\myvec{-\frac{1}{2}\\-\frac{1}{2}\\1},
\vec{u}_2=\myvec{1\\-1\\0},
\vec{u}_3=\myvec{-1\\-1\\1}
\intertext{Normalizing the eigen vectors we get,}
\vec{u}_1=\myvec{\frac{-1}{\sqrt{6}}\\\frac{-1}{\sqrt{6}}\\\sqrt{\frac{2}{3}}},
\vec{u}_2=\myvec{\frac{1}{\sqrt{2}}\\-\frac{1}{\sqrt{2}}\\0},
\vec{u}_3=\myvec{-\sqrt{\frac{1}{3}}\\-\sqrt{\frac{1}{3}}\\\sqrt{\frac{1}{3}}}
\end{align}
Hence we obtain $\vec{U}$ of \eqref{eq:solutions/3/6/eqSVD} as follows,
\begin{align}
\vec{U}=\myvec{\frac{-1}{\sqrt{6}}&\frac{1}{\sqrt{2}}&-\sqrt{\frac{1}{3}}\\\frac{-1}{\sqrt{6}}&\frac{-1}{\sqrt{2}}&-\sqrt{\frac{1}{3}}\\\sqrt{\frac{2}{3}}&0&\sqrt{\frac{1}{3}}}\label{eq:solutions/3/6/eqU}
\end{align}
After computing the singular values from eigen values $\lambda_1, \lambda_2, \lambda_3$ we get $\vec{S}$ of \eqref{eq:solutions/3/6/eqSVD} as follows,
\begin{align}
\vec{S}=\myvec{\frac{1}{\sqrt{3}}&0\\0&1\\0&0}\label{eq:solutions/3/6/eqS}
\end{align}
Now, calculating eigen value of $\vec{M}^T\vec{M}$,
\begin{align}
\mydet{\vec{M}^T\vec{M} - \lambda\vec{I}} &= 0\\
\implies\myvec{2-\lambda&1\\1&2-\lambda} &=0\\
\implies\lambda^2-4\lambda+3 &=0
\end{align}
Hence eigen values of $\vec{M}^T\vec{M}$ are,
\begin{align}
\lambda_4 &= 3\\
\lambda_5 &= 1
\end{align}
Hence the eigen vectors of $\vec{M}^T\vec{M}$ are,
\begin{align}
\vec{v}_1=\myvec{-1\\-1},
\vec{v}_2=\myvec{1\\-1}
\intertext{Normalizing the eigen vectors we get,}
\vec{v}_1=\myvec{\frac{2}{\sqrt{5}}\\\frac{1}{\sqrt{5}}},
\vec{v}_2=\myvec{-\frac{1}{\sqrt{5}}\\\frac{2}{\sqrt{5}}}
\end{align}
Hence we obtain $\vec{V}$ of \eqref{eq:solutions/3/6/eqSVD} as follows,
\begin{align}
\vec{V}=\myvec{\frac{-1}{\sqrt{2}}&\frac{1}{\sqrt{2}}\\\frac{-1}{\sqrt{2}}&\frac{-1}{\sqrt{2}}}
\end{align}
Finally from \eqref{eq:solutions/3/6/eqSVD} we get the Singualr Value Decomposition of $\vec{M}$ as follows,
\begin{align}
\vec{M} = \myvec{\frac{-1}{\sqrt{6}}&\frac{1}{\sqrt{2}}&-\sqrt{\frac{1}{3}}\\\frac{-1}{\sqrt{6}}&\frac{-1}{\sqrt{2}}&-\sqrt{\frac{1}{3}}\\\sqrt{\frac{2}{3}}&0&\sqrt{\frac{1}{3}}}\myvec{\frac{1}{\sqrt{3}}&0\\0&1\\0&0}\myvec{\frac{-1}{\sqrt{2}}&\frac{1}{\sqrt{2}}\\\frac{-1}{\sqrt{2}}&\frac{-1}{\sqrt{2}}}^T
\end{align}
Now, Moore-Penrose Pseudo inverse of $\vec{S}$ is given by,
\begin{align}
\vec{S_+} = \myvec{\frac{1}{\sqrt{3}}&0&0\\0&1&0}
\end{align}
From \eqref{eq:solutions/3/6/eqX} we get,
\begin{align}
\vec{U}^T\vec{b}&=\myvec{-\frac{3}{\sqrt{6}}\\\frac{5}{\sqrt{2}}\\0}\\
\vec{S_+}\vec{U}^T\vec{b}&=\myvec{-\frac{3}{\sqrt{18}}\\\frac{5}{\sqrt{2}}}\\
\vec{x} = \vec{V}\vec{S_+}\vec{U}^T\vec{b} &= \myvec{3\\-2}\label{eq:solutions/3/6/eqXSol1}
\end{align}
Verifying the solution of \eqref{eq:solutions/3/6/eqXSol1} using,
\begin{align}
\vec{M}^T\vec{M}\vec{x} = \vec{M}^T\vec{b}\label{eq:solutions/3/6/eqVerify}
\end{align}
Evaluating the R.H.S in \eqref{eq:solutions/3/6/eqVerify} we get,
\begin{align}
\vec{M}^T\vec{M}\vec{x} &= \myvec{4\\-1}\\
\implies\myvec{2&1\\1&2}\vec{x} &= \myvec{4\\-1}\label{eq:solutions/3/6/eqMateq}
\end{align}
Solving the augmented matrix of \eqref{eq:solutions/3/6/eqMateq} we get,
\begin{align}
\myvec{2&1&4\\1&2&-1}\xleftrightarrow[R_2\rightarrow{R_2-2R_1}]{R_2\rightarrow{R_1}}\myvec{1&2&-1\\0&-3&6}\\\xleftrightarrow[R_1\leftarrow{R_1-2R_2}]{R_2\leftarrow{R_2/-3}}\myvec{1&0&3\\0&1&-2}
\end{align}
Hence, Solution of \eqref{eq:solutions/3/6/eqVerify} is given by,
\begin{align}
\vec{x}=\myvec{3\\-2}\label{eq:solutions/3/6/eqX2}
\end{align}
Comparing results of $\vec{x}$ from \eqref{eq:solutions/3/6/eqXSol1} and \eqref{eq:solutions/3/6/eqX2} we conclude that the solution is verified.

