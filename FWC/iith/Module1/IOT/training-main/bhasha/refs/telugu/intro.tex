ఈ దస్తూరి లో నేను \LaTeX లో తెలుగు వ్రాయడానికి వీలుగా తయారు చేసుకున్నాను. కేవలం ఒక ఖని మాత్రమే కాకుండా బోళ్ళన్ని ఖనులని ఈదస్తూరిలో ఉంచచ్చు. 
Default ఖని ని నిర్ణయించాలి అంటే, ముందు ఉంచబడ్డ  setmainfont LaTeX
ఆదేసాలలో ఒక దానిని అమలు చేస్తే సరిపోతుంది.
ఇంక వేరు వేరు ఖనులను  అమలు చేయాలి అంటే, కింద చూపించిన ప్రకారముగా వ్రాసుకుంటే సరిపోతుంది.~\\

\begin{center}
\Large{

\akshar{అక్షర్}\\


\dhurjati{ధూర్జటి}\\


\gautami{గౌతమి}\\


\gidugu{గిడుగు}\\


\gurajada{గురజాడ}\\


\lakkireddy{లక్కి రెడ్డి}\\


\lohit{లోహిత్}\\


\mallanna{మల్లన్న - ప్రస్తుతం నాకు అన్నిటికన్నా ప్రియమైనది}\\


\mandali{మండలి}\\


\nats{నాట్స్}\\


\ntr{ఎన్.టీ.ఆర్}\\


\peddana{పెద్దన}\\


\ponnala{పొన్నాల}\\


\pothana{పోతన}\\


\psr{పొట్టి శ్రీరాములు}\\


\ramabhadra{రామభద్ర}\\


\ramaraja{రామరాజ}\\


\raviprakash{రవి ప్రకాష్}\\


\skdr{శ్రీకృష్ణదేవరాయ}\\


\suranna{సూరన్న}\\


\suravaram{సురవరం}\\


\syr{శ్యామలా రమణ}\\


\trk{తెనాలి రామకృష్ణ}\\


\timmana{తిమ్మన}\\


\vemana{వేమన}\\


\notosans{నోటో-సాంస్}\\


\notosansUI{నోటో-సాంస్-యూఐ}\\


\notoserif{నోటో-సెరిఫ్}\\
}
\end{center}

ఈ ఖనులు అన్నీ స్వేచ్చా సాఫ్ట్-వేర్ నించి, మరియు సిలికానాంధ్రా అంతర్జాల పేజీ నించీ దిగుమతి చేయడము అయినది. 
ఈ దస్తూరి పనితీరు మీద ఏటువంటి హామీ ఇవ్వబడుట లేదు.


ఒవర్-లీఫ్ లో ఈ దస్తువు పని చేయాలి అంటే, XeLaTeX ని compiler
 గా నియమించ వలయును.

