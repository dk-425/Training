\renewcommand{\theequation}{\theenumi}
\begin{enumerate}[label=\thesubsection.\arabic*.,ref=\thesubsection.\theenumi]
\numberwithin{equation}{enumi}

\item Revisiting Problem \eqref{prob:line_gram_schmidt},
%
\begin{align}
\label{eq:decomp_qr_given}
{\alpha} = \myvec{3\\-1\\0},
 {\beta} = \myvec{2\\1\\-3}
\end{align}
%
we can express
\begin{align}
\begin{split}
\alpha = k_1\vec{u}_1 
\\
\beta = r_1\vec{u}_1 +k_2\vec{u}_2
\end{split}
\label{eq:decomp_gram}
\end{align}
%
where 
\begin{align}
k_1 = \norm{\alpha}, \vec{u}_1 = \frac{\vec{\alpha}}{k_1} 
\\
r_1 = \frac{\vec{u}_1^T\beta}{\norm{\vec{u}_1}^2}, 
\vec{u}_2 = \frac{\beta - r_1 \vec{u}_1}{\norm{\beta - r_1 \vec{u}_1}}
\\
k_2 = {\vec{u}_2^T\beta}
\end{align}
From \eqref{eq:decomp_gram}, 
\begin{align}
\myvec{\alpha & \beta } = \myvec{\vec{u}_1 & \vec{u}_2}\myvec{k_1 & r_1 \\ 0 & k_2} 
\end{align}
%
This is known as $\vec{Q}\vec{R}$ decomposition, where 
\begin{align}
\vec{R} = \myvec{k_1 & r_1 \\ 0 & k_2} 
\\
\vec{Q} = \myvec{\vec{u}_1 & \vec{u}_2}
\end{align}
%
Note that $\vec{R}$ is an upper triangular matrix and 
\begin{align}
\vec{Q}^T\vec{Q} = \vec{I}.
\end{align}
\item From \eqref{eq:decomp_qr_given},
\begin{align}
k_1 = \sqrt{10}, \vec{u}_1 = \frac{1}{\sqrt{10}} \myvec{3\\-1\\0},
\\
r_1 = \frac{1}{2}, \vec{u}_2 = \frac{1}{\sqrt{46}}\myvec{1\\3\\-6}
\\
k_2 = \sqrt{\frac{23}{2}}
\end{align}
Thus, we obtain the $\vec{Q}\vec{R}$ decompositon
\begin{align}
\myvec{
3 & 2
\\
-1 & 1
\\
0 & -3
}
=\myvec{
\frac{3}{\sqrt{10}} & \frac{1}{\sqrt{46}}
\\
\frac{-1}{\sqrt{10}} & \frac{3}{\sqrt{46}}
\\
0 & \frac{-6}{\sqrt{46}}
}
\myvec{
\sqrt{10}& \frac{1}{2}
\\
0&\sqrt{\frac{23}{2}}
}
\end{align}

\end{enumerate}
