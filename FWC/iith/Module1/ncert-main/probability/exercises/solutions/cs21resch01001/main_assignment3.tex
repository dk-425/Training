

\title{Assignment 3 -Probability and Random Variable}
\author{seshikanth CS21RESCH01001\thanks{GVV Sharma sir}}
\date{\today}

\documentclass[journal,12pt,twocoloums]{IEEEtran}
\usepackage{amsthm}
\usepackage{graphicx}
\usepackage{mathrsfs}
\usepackage{txfonts}
\usepackage{stfloats}
\usepackage{pgfplots}
\usepackage{cite}
\usepackage{cases}
\usepackage{mathtools}
\usepackage{caption}
\usepackage{enumerate}	
\usepackage{enumitem}
\usepackage{amsmath}
\usepackage[utf8]{inputenc}
\usepackage[english]{babel}
\usepackage{multicol}
%\usepackage{xtab}
\usepackage{longtable}
\usepackage{multirow}
%\usepackage{algorithm}
%\usepackage{algpseudocode}
\usepackage{enumitem}
\usepackage{mathtools}
\usepackage{gensymb}
\usepackage{hyperref}
%\usepackage[framemethod=tikz]{mdframed}
\usepackage{listings}
    %\usepackage[latin1]{inputenc}                                 %%
    \usepackage{color}                                            %%
    \usepackage{array}                                            %%
    \usepackage{longtable}                                        %%
    \usepackage{calc}                                             %%
    \usepackage{multirow}                                         %%
    \usepackage{hhline}                                           %%
    \usepackage{ifthen}                                           %%
  
 \begin{document}
 \maketitle
 

\title{Assignment 3 -Probability and Random Variable}
\author{seshikanth CS21RESCH01001\thanks{GVV Sharma sir}}
\date{\today}

 \textbf{Problem Statement prob 3.2}-An experiment succeeds twice as often
as it fails. Find the probability that in the next six trials, there will be at least 4 successes.\\
\textbf{Solutions:}\\

Success: Failure ratio :2:3

It means prob of success = p = $\frac{2}{3}$, 

Prob of failture = q  = $\frac{1}{3}$

Let X be total number of successes in n trails, n = 6

$Pr(X \geq 4) = { 6 \choose 4} p^{4} q^2 + { 6 \choose 5} p^{5} q + { 6 \choose 6} p^{6} q^0,  \\
              = \textbf{0.6804},  \textbf{is the required probability}
$

Type of distribution = binomial(n,p) = binomial(6,$\frac{2}{3}$)
\end{document}
        