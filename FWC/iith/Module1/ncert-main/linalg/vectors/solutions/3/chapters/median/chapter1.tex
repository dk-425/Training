\renewcommand{\theequation}{\theenumi}
\begin{enumerate}[label=\thesection.\arabic*.,ref=\thesection.\theenumi]
\numberwithin{equation}{enumi}
\item Let the medians $BE$ and $CF$ in Fig. \ref{fig:3.12.3_ch1_two_median} intersect at $O$, such that
\begin{equation}
\begin{split}
\frac{OB}{OE} &= k_1
\\
\frac{OC}{OF} &= k_2
\end{split}
\end{equation}
%Then  $k_1 = k_2 = 2$.
%
\begin{figure}[!h]
\centering
\resizebox {\columnwidth} {!} {
\begin{tikzpicture}
  [
    scale=2,
    >=stealth,
    point/.style = {draw, circle,  fill = black, inner sep = 0.5pt},
    dot/.style   = {draw, circle,  fill = black, inner sep = .2pt},
  ]
  \coordinate [point, label={below left:$B$ $(0, 0)$}] (B) at (0, 0);
    \node (A) at +(60:{2*sqrt(3)}) [point, label = above:$A$ ${(a,b)}$  ] {};
  \coordinate [point, label={below left:$(c,0)$ $C$ }] (C) at ($ (3,0) + sqrt(3)*(1,0) $);

  \draw  (A) -- (C) -- (B) -- (A);
  \node (E) at ($(A)!0.5!(C)$) [point, label = {right:$E$}]{};
  \node (F) at ($(A)!0.5!(B)$) [point, label = {left:$F$}]{};
  \path
     (B)    edge  node[sloped, anchor=center, below, text width=2.0cm] { $k_1:1$}     (E)  
	 (C)    edge  node[sloped, anchor=east, below, text width=2.0cm] { $1:k_2$}     (F);
  \node (O) at ($(B)!0.67!(E)$) [point, label = {below:$O$}]{};  
\end{tikzpicture}


}
\caption{Medians $BE$ and $CF$}
\label{fig:3.12.3_ch1_two_median}
\end{figure}
%Let the coordinates of $A$, $B$ and $C$ be $\brak{a,b}$, $\brak{0,0}$ and $\brak{c,0}$ respectively. 
Using \eqref{eq:line_section_form},
%
\begin{align}
E &= \frac{\vec{A}+\vec{C}}{2} 
\\
F &= \frac{\vec{A}+\vec{B}}{2} 
\label{eq:3.12.3_ch1_ratio_ef}
\end{align}
%
Similarly, since $O$ divides $BE$ in the ratio $k_1:1$ and $CF$ in the ratio $k_2:1$.
 %
\begin{align}
O = \frac{k_1\vec{E}+\vec{B}}{k_1+1} &=  \frac{k_2\vec{F}+\vec{C}}{k_2+1} 
\\
\implies \frac{k_1\brak{\frac{\vec{A}+\vec{C}}{2}} +B}{k_1+1} &=  \frac{k_2\brak{\frac{\vec{A}+\vec{B}}{2} }+C}{k_2+1} 
\label{eq:3.12.3_ch1_ratio_2}
\end{align}
upon substituting from \eqref{eq:3.12.3_ch1_ratio_ef}.
Simplifying \eqref{eq:3.12.3_ch1_ratio_2},
\begin{align}
\frac{k_1\brak{\vec{A}+\vec{C}} +2\vec{B}}{k_1+1} =  \frac{k_2\brak{\vec{A}+\vec{B}}+2\vec{C}}{k_2+1} 
\end{align}
which can be expressed as
\begin{multline}
\implies \sbrak{k_1\brak{k_2+1}-k_2\brak{k_1+1}}\vec{A}
\\
 +\sbrak{2\brak{k_2+1}-k_2\brak{k_1+1}}\vec{B}
\\ +  \sbrak{k_1\brak{k_2+1} -2\brak{k_1+1}}\vec{C} = 0
\end{multline}
resulting in 
\begin{align}
\vec{B} = \frac{\brak{k_1-k_2}\vec{A}+\brak{k_1k_2 -k_1 -2}}{k_1k_2 -k_2 -2}
\end{align}
%
If the above equation has a solution, then $\vec{A}, \vec{B}$ and $\vec{C}$ lie on a straight line.  Since that is not the case, the only possibility is 
\begin{align}
k_1-k_2 &= 0
\\
k_1k_2 -k_1 -2 &= 0
\\
k_1k_2 -k_2 -2 &= 0
\\
\implies k_1=k_2&=2
\end{align}
{\em If $\vec{A}, \vec{B}, \vec{C}$ lie on a triangle, they are linearly independent.}    In which case, 
\begin{align}
x_1\vec{A}+ x_2\vec{B}+x_3\vec{C} &= 0
\\
\implies x_1 = x_2=x_3 = 0,
\end{align}
Else, they are linearly dependent and lie on a straight line.
\item In Fig. \ref{fig:3.12.3_ch1_two_median},
\begin{align}
\vec{E} &=  \frac{\vec{A}+\vec{C}}{2} \quad \text{and}
\\
\vec{O}&= \frac{\vec{B}+2\vec{E}}{3}
\\
&= \frac{\vec{A}+\vec{B}+\vec{C}}{3}
\end{align}
\end{enumerate}
	
