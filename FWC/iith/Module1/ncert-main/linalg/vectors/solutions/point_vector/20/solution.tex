Let the points be $\vec{P}$=\myvec{ 1 \\ -1 \\ 2 }, $\vec{Q}$=\myvec{ 3 \\ 4 \\ -2}, $\vec{R}$=\myvec{ 0 \\ 3 \\ 2 } and $\vec{S}$=\myvec{ 3 \\ 5 \\ 6}.
The direction vector for the line through the points $\vec{P}$ and $\vec{Q}$ is
\begin{align}
\vec{A} &= \vec{P}-\vec{Q} \\
\implies\vec{A} & = \myvec{ 1 \\ -1 \\ 2 } - \myvec{ 3 \\ 4 \\ -2} \\
\implies\vec{A} &= \myvec{ -2 \\ -5 \\ 4}
\end{align}
The direction vector for the line through the points $\vec{R}$ and $\vec{S}$ is
\begin{align}
\vec{B} &= \vec{R}-\vec{S} \\
\implies\vec{B} & = \myvec{ 0 \\ 3 \\ 2 } - \myvec{ 3 \\ 5 \\ 6} \\
\implies\vec{B} &= \myvec{ -3 \\ -2 \\ -4}\\
\end{align}
To check if the two lines are perpendicular, we perform scalar product of the two direction vectors $\vec{A}$ and $\vec{B}$ as follows
\begin{align}
\vec{A}\vec{B} &=  \vec{A}^T\vec{B}\\
&= \myvec{ -2 & -5 & 4}\myvec{ -3 \\ -2 \\ -4}\\
&= 6+10-16\\
&= 0
\end{align}

Thus, the lines are \textbf{perpendicular}. 
