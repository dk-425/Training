\renewcommand{\theequation}{\theenumi}
\begin{enumerate}[label=\arabic*.,ref=\thesubsection.\theenumi]
\numberwithin{equation}{enumi}
\item 	In Fig. \ref{fig:tri_sum_angle}, the sum of all the angles on the top or bottom side of the straight line $XY$ is $180\degree$.


\begin{figure}[!ht]
	\begin{center}
		\resizebox{\columnwidth}{!}{%Code by GVV Sharma
%December 6, 2019
%released under GNU GPL
%Sum of the angles of a right angled triangle 

\begin{tikzpicture}
[scale=2,>=stealth,point/.style={draw,circle,fill = black,inner sep=0.5pt},]

%Triangle sides
\def\a{4}
\def\c{3}

%Section Ratio
\def\k{1.2}


%Labeling points
\node (A) at (0,\c)[point,label=above right:$A$] {};
\node (B) at (0, 0)[point,label=below left:$B$] {};
\node (C) at (\a, 0)[point,label=below right:$C$] {};

%Translating coordinates
\node (Y) at ($(A) + (1,0)$)[point,label=above right:$Y$] {};
\node (X) at ($(A) - (1,0)$)[point,label=above right:$X$] {};
\node (T) at ($(A) + (0,1)$)[point,label=above right:$T$] {};

%Section formula
\node (V) at ($ (C)!\k!(A) $)[point,label=above right:$V$] {};

%Drawing triangle ABC
\draw (A) -- node[left] {$\textrm{c}$} (B) -- node[below] {$\textrm{a}$} (C) -- node[above,xshift=2mm] {$\textrm{b}$} (A);

%Joining other points
\draw (Y)--(A);
\draw (X)--(A);
\draw (T)--(A);
\draw (V)--(A);

%Drawing and marking angles
\tkzMarkAngle[fill=orange!40,size=0.5cm,mark=](A,C,B)
\tkzMarkAngle[fill=orange!40,size=0.5cm,mark=](V,A,X)
\tkzMarkRightAngle[fill=blue!20,size=.3](A,B,C)
\tkzMarkRightAngle[fill=blue!20,size=.3](B,A,X)
\tkzLabelAngle[pos=0.65](A,C,B){$\theta$}
\tkzLabelAngle[pos=0.65](V,A,X){$\theta$}

\end{tikzpicture}
}
	\end{center}
	\caption{Sum of angles of a triangle}
	\label{fig:tri_sum_angle}	
\end{figure}



\item
In Fig. \ref{fig:tri_sum_angle}, the straight line making an angle of $90\degree$ to the side $AB$ is said to be parallel to the side $BC$. Note there is an angle at $A$ that is equal to $\theta$.  This is one property of parallel lines.  Thus, $\angle YAZ = 90\degree$.


\item
	Show that $\angle VAZ = 90\degree - \theta$
		
	\solution Considering the line $XAZ$,
	\begin{align}
	\theta + 90\degree + \angle VAZ &= 180\degree \\
	\Rightarrow  \angle VAZ =  90\degree - \theta
	\end{align}

\item
	\label{prob:tri_compl_angle}
	Show that $\angle BAC = 90\degree - \theta$.
	
	\solution Consider the line $VAB$ and and use the approach in the previous problem.  Note that this implies that $\angle VAZ = \angle BAC$.  Such angles are known as vertically opposite angles. 
	 
\item
Sum of the angles of a triangle is equal to $180\degree$.
%
\item Draw Fig. \ref{fig:tri_sum_angle} for $a = 4, c =3$.
%
\\
\solution Problem \ref{const:tri_right_angle} is used to draw $\triangle ABC$.  The remaining points are obtained as

\begin{align}
\vec{Y} &= \vec{A} + \myvec{1\\0} = \myvec{1\\3}
\\
\vec{X} &= \vec{A} - \myvec{1\\0} = \myvec{-1\\3}
\\
\vec{T} &= \vec{A} + \myvec{0\\1} = \myvec{0\\4}
\end{align}
%
and 
\begin{align}
\frac{VC}{AC} &= k+1
\\
\label{eq:tri_section_formula}
\implies \vec{A} &= \frac{k\vec{C} + \vec{V}}{k+1}  
\\
\implies \vec{V} &= \frac{\brak{k+1}\vec{A} - k\vec{C}}{k}  
\end{align}
%
for $k = 0.2$. \eqref{eq:tri_section_formula} is known as the {\em section formula}.
%
The python code for  Fig. \ref{fig:tri_sum_angle} is
\begin{lstlisting}
codes/triangle/tri_sum_angle.py
\end{lstlisting}
%
and the equivalent latex-tikz code is
%
\begin{lstlisting}
figs/triangle/tri_sum_angle.tex
\end{lstlisting}
\end{enumerate}
\subsection{Baudhayana Theorem}
Use Fig. \ref{fig:tri_baudh} for all problems in this section.
\renewcommand{\theequation}{\theenumi}
\begin{enumerate}[label=\arabic*.,ref=\thesubsection.\theenumi]
\numberwithin{equation}{enumi}



\item  Show that
	\begin{equation}
	\cos \theta = \sin \brak{90\degree - \theta}
	\end{equation}


\begin{figure}[!ht]
	\begin{center}
		\resizebox{\columnwidth}{!}{%Code by GVV Sharma
%December 7, 2019
%released under GNU GPL
%Proof of Baudhyana Theorem

\begin{tikzpicture}
[scale=2,>=stealth,point/.style={draw,circle,fill = black,inner sep=0.5pt},]

%Triangle sides
\def\a{4}
\def\c{3}
\def\b{sqrt(\a^2+\c^2)}

%Trigonometric ratios
\def\ct{\a/\b}
\def\st{\c/\b}

%perp distance
\def\r{\a*\st}

%Section Ratio
\def\k{1.2}


%Labeling points
\node (A) at (0,\c)[point,label=above right:$A$] {};
\node (B) at (0, 0)[point,label=below left:$B$] {};
\node (C) at (\a, 0)[point,label=below right:$C$] {};

%Foot of perpendicular

\node (D) at ($({\r*\st}, {\r*\ct})$)[point,label=above right:$D$] {};


%Drawing triangle ABC
\draw (A) -- node[left] {$\textrm{c}$} (B) -- node[below] {$\textrm{a}$} (C) -- node[above,xshift=2mm] {$\textrm{b}$} (A);

%Joining BD
\draw (B)--(D);

%Drawing and marking angles
\tkzMarkAngle[fill=orange!40,size=0.5cm,mark=](A,C,B)
\tkzMarkAngle[fill=orange!40,size=0.4cm,mark=](D,B,A)
\tkzMarkAngle[fill=green!40,size=0.5cm,mark=](B,A,C)
\tkzMarkAngle[fill=green!40,size=0.5cm,mark=](C,B,D)
\tkzMarkRightAngle[fill=blue!20,size=.2](A,B,C)
\tkzMarkRightAngle[fill=blue!20,size=.2](B,D,A)
\tkzLabelAngle[pos=0.65](A,C,B){$\theta$}
\tkzLabelAngle[pos=0.65](A,B,D){$\theta$}
\tkzLabelAngle[pos=1](B,A,C){\rotatebox{-45}{$\alpha = 90\degree -\theta$}}
\tkzLabelAngle[pos=0.65](C,B,D){$\alpha$}

\end{tikzpicture}
}
	\end{center}
	\caption{Baudhayana Theorem}
	\label{fig:tri_baudh}	
\end{figure}


\solution From Problem \ref{prob:tri_compl_angle} and  \eqref{eq:tri_trig_defs}
%
\begin{multline}
\label{eq:tri_90-ang}
\cos \angle BAC = \cos \alpha =	\cos \brak{90\degree-\theta} = \frac{c}{b} 
\\
= \sin \angle ABC = \sin \theta
\end{multline}
%
\item
Show that 
%
\begin{equation}
\label{ch1_budh_basic}
b = a \cos \theta + c \sin \theta
\end{equation}
%
\solution We observe that
%
\begin{align}
BD &= a \cos \theta \\
AD &= c \cos\alpha = c \sin \theta \quad \brak{\text{From} \quad \eqref{eq:tri_90-ang}
}
\end{align}
%
Thus,
\begin{equation}
BD + AD = b = a \cos \theta + c \sin \theta
\end{equation}
\item
From \eqref{ch1_budh_basic}, show that
%
\begin{equation}
%
\label{eq:tri_sin_cos_id}
\sin ^2 \theta + \cos ^2 \theta = 1
\end{equation}


%
\solution Dividing both sides of \eqref{ch1_budh_basic} by $b$, 
\begin{align}
1 &= \frac{a}{b}\cos\theta + \frac{c}{b}\sin\theta\\
\Rightarrow &\sin ^2 \theta + \cos ^2 \theta = 1 \quad \brak{\text{from} \quad \eqref{eq:tri_trig_defs}}
\end{align}

\item
	Using \eqref{ch1_budh_basic}, show that
	\begin{equation}
	\label{eq:tri_baudh}
	c^2 = a^2 + b^2
	\end{equation}
	\eqref{eq:tri_baudh} is known as the Baudhayana theorem.  It is also known as the Pythagoras theorem.

\solution From \eqref{ch1_budh_basic},
\begin{align}
c &= a\frac{a}{c} + b \frac{b}{c} \quad \brak{\text{from} \quad \eqref{eq:tri_trig_defs}}\\
\Rightarrow c^2 &= a^2 + b^2
\end{align}
%
%
\item Draw Fig. \ref{fig:tri_baudh} for $a = 4, c =3$.
\label{const:tri_baudh}
%
\\
\solution Problem \ref{const:tri_right_angle} is used to draw $\triangle ABC$.
%
Using Problem \ref{prob:tri_polar},
\begin{align}
\vec{D} &= BD\myvec{\cos \alpha\\  \sin \alpha} 
&= a \sin \theta \myvec{ \sin \theta \\ \cos \theta } 
\label{eq:tri_baudh_foot}
\end{align}
%
Using \eqref{eq:tri_baudh_foot}, the python code for  Fig. \ref{fig:tri_baudh} is
\begin{lstlisting}
codes/triangle/tri_baudh.py
\end{lstlisting}
%
and the equivalent latex-tikz code is
%
\begin{lstlisting}
figs/triangle/tri_baudh.tex
\end{lstlisting}
%
\item Using 	\eqref{eq:tri_baudh}, for $a = 4, c = 3$,
%
\begin{align}
b = \sqrt{a^2+c^2} = \sqrt{4^2+3^2} = 5
\end{align}
%
\item For  point $\vec{D} = \myvec{d_1\\d_2}$, its {\em norm} is defined as
%
\begin{align}
OD = d_1^2+d_2^2 = \norm{\vec{D}} \define \sqrt{\vec{D}^T\vec{D}}, 
\label{eq:tri_norm_def}
\end{align}
%
where 
%
\begin{align}
\label{eq:tri_transpose_def}
 \vec{D}^T  \define \myvec{d_1 & d_2},
\\
\vec{D}^T\vec{D} \define \myvec{d_1 & d_2} \myvec{d_1 \\ d_2} = d_1^2+d_2^2
\end{align}
%
\eqref{eq:tri_transpose_def} is the definition of {\em transpose}. $\vec{D}$ is defined to be a {\em column vector} and $\vec{D}^T$  is the corresponding {\em row vector} representing the same point.

\item Also, it is easy to verify that
%
\begin{align}
\label{eq:tri_norm_dist}
AC \define  \norm{\vec{A}-\vec{C}} =  \norm{\myvec{4\\-3}} = \sqrt{3^2+4^2} = 5
\end{align}
%
This is known as the {\em distance formula}.
%
\item Prove the distance formula in \eqref{eq:tri_norm_dist} using Baudhayana theorem.

\end{enumerate}
