
\begin{enumerate}
  \item \label{prob:solutions/matrix/38/Problem:1}
\begin{align}
A = \myvec{\cos{\alpha} && \sin{\alpha} \\ $-$\sin{\alpha} && \cos{\alpha}} \\[1em]
\implies A^T = \myvec{\cos{\alpha} && $-$\sin{\alpha} \\ \sin{\alpha} && \cos{\alpha}}    
\end{align}
%Need to prove that $A^T$A = I (Identity Matrix)\\\\
%For this we express each of the matrices as a complex number to find the product of those two matrices $A^T$ and A.\\\\
%We know that a complex number \myvec{a1 \\ a2} can be represented as
\begin{align}
\myvec{a_1 \\ a_2} = \myvec{a_1 && $-$a_2 \\ a_2 && a_1}\myvec{1 \\ 0}\\[1em]
i.e \ A = \myvec{\cos{\alpha} \\ $-$\sin{\alpha}}\\[1em]
\implies \ A = e^{-j\alpha}\\[1em]
i.e. \ A^T = \myvec{\cos{\alpha} \\ \sin{\alpha}} \\[1em]
\implies \ A^T = e^{j\alpha}\\[1em]
%\implies A^TA = \myvec{\frac{e^{j\alpha}+e^{-j\alpha}}{2} \\[0.5em] \frac{e^{j\alpha}-e^{-j\alpha}}{2j}}\myvec{\frac{e^{j\alpha}+e^{-j\alpha}}{2} \\[0.5em] $-$\frac{e^{j\alpha}-e^{-j\alpha}}{2j}}\\[1em]
\implies A^TA = e^{j\alpha}e^{-j\alpha} = 1\\[1em]
%\implies A^TA = \myvec{1&&0\\0&&1}\myvec{1\\0}\\[1em]
\implies A^TA = I
\end{align} \\
\item \label{prob:solutions/matrix/38/Problem:2}
\begin{align}
A = \myvec{\sin{\alpha} && \cos{\alpha} \\ $-$\cos{\alpha} && \sin{\alpha}} \\[1em]
\implies A^T = \myvec{\sin{\alpha} && $-$\cos{\alpha} \\ \cos{\alpha} && \sin{\alpha}}    
\end{align}\\
%Need to prove that $A^T$A = I (Identity Matrix)\\\\
%For this we express each of the matrices as a complex number to find the product of those two matrices $A^T$ and A.\\\\
Using (3),
\begin{align}
i.e \ A = \myvec{\sin{\alpha} \\ $-$\cos{\alpha}}\\[1em]
\implies \ A = e^{-j\brak{\frac{n\pi}{2}-\alpha}}\label{prob:solutions/matrix/38/equation:13}\\[1em]
A^T = \myvec{\sin{\alpha} \\ \cos{\alpha}} \\[1em]
\implies \ A^T = e^{j\brak{\frac{n\pi}{2}-\alpha}}\label{prob:solutions/matrix/38/equation:15}\\[1em]
%\implies A^TA = \myvec{\frac{e^{j\alpha}-e^{-j\alpha}}{2j} \\[0.5em] \frac{e^{j\alpha}+e^{-j\alpha}}{2}}\myvec{\frac{e^{j\alpha}-e^{-j\alpha}}{2j} \\[0.5em] $-$\frac{e^{j\alpha}+e^{-j\alpha}}{2}} \\[1em]
%\implies A^TA = \myvec{1&&0\\0&&1}\myvec{1\\0}\\[1em]
\implies A^TA = e^{j\brak{\frac{n\pi}{2}-\alpha}}e^{-j\brak{\frac{n\pi}{2}-\alpha}} = 1 \\[1em]
\implies A^TA = I
\end{align}\\
Here, n in \eqref{prob:solutions/matrix/38/equation:13} and \eqref{prob:solutions/matrix/38/equation:15} is an odd number.\\\\
%Using (3), the expressions in equation (6) and equation (12) were simplified to obtain I.\\\\
Hence proved for both Problems \ref{prob:solutions/matrix/38/Problem:1} and \ref{prob:solutions/matrix/38/Problem:2}.
\end{enumerate}
