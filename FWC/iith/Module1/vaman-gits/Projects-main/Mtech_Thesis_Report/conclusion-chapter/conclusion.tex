\chapter{Conclusion and Future Directions}\label{chap:conclude}
\subsubsection{Conclusion}
\begin{itemize}
 \item The basic principle behind motor control, use of serial interfaces (UART, SPI, I2C) and the inherent software remains similar to that of bigger sophisticated hardware of actual ground/aerial vehicle.
	\item Thus, The applications implemented on the UGV/UAV using ESP32/VAMAN can be scaled up to work on a real-life ground/aerial vehicle. This makes the UGV/UAV kit a good low cost prototype hardware.
	\item The 3GPP standards for Non-Terrestrial Network for 5G are only a couples of months away from their release, Thus the SATCOM infrastructure for 5G is not far from being commercial. 
	\item The experimental setup and results shown in this presentation confirms basic compatibility between satellite and 5G system.
\end{itemize}
\subsubsection{Future Directions}
\begin{itemize}
	\item The SATCOM setup discussed in this report doesn't has a operational PHY layer. The future task would be to test the network working by including the PHY layer. 
	\item Since, the Uu interface between the UE and gNB is replaced by a satellite link in a transparent architecture, the 5G PHY layer needs to be modified to cater the satellite transmission and reception.
	\item Refering to the navigation part of this report, the next step would be to scale the application implemented on UGV/UAV hardware to real ground vehicles.
\end{itemize}