\renewcommand{\theequation}{\theenumi}
\begin{enumerate}[label=\arabic*.,ref=\thesubsection.\theenumi]
\numberwithin{equation}{enumi}

\item Solve
\begin{align}
\min_{\vec{x}} Z &= \myvec{3 & 2}\vec{x}
\\
s.t. \quad 
\myvec{
-1 & -1
\\
3 & 5
}
\vec{x} &\preceq \myvec{-8\\15}
\\
\vec{x} &\succeq \vec{0}
\end{align}
\item Solve
\begin{align}
\min_{\vec{x}} Z &= \myvec{200 & 500}\vec{x}
\\
s.t. \quad 
\myvec{
-1 & -2
\\
3 & 4
}
\vec{x} &\preceq \myvec{-10\\24}
\\
\vec{x} &\succeq \vec{0}
\end{align}
\item Maximise Z=3x+4y\\
subject to the constraints : x+y$\leq$4, x$\geq$0, y$\geq$ 0.\\
\item Minimise Z=-3x+4y\\
subject to x+2y$\leq$8, 3x+2y$\leq$12, x$\geq$0, y$\geq$0.\\
\item Maximise Z=5x+3y
subject to 3x+5y$\leq$15, 5x+2y$\leq$10, x$\geq$0, y$\geq$0.\\
\item Minimise Z=3x+5y
such that x+3y$\geq$3, x+y$\geq$2, x,y$\geq$0.\\
\item Maximise Z=3x+2y
subject to x+2y$\leq$10, 3x+y$\leq$15, x,y$\geq$0.\\
\item Minimise Z=x+2y
subject to 2x+y$\geq$3, x+2y$\geq$6, x,y$\geq$0.\\
Show that the minimum of Z occurs at more than two points.\\
\item Minimise and Maximise Z=5x+10y
subject to x+2y$\leq$120, x+y$\geq$60, x-2y$\geq$0, x,y$\geq$0.\\
\item Minimise and Maximise Z=x+2y
subject to x+2y$\geq$100, 2x-y$\leq$0, 2x+y$\leq$200; x,y$\geq$0.\\
\item Maximise Z=-x+2y, subject to the constraints:
x$\geq$3, x+y$\geq$5, x+2y$\geq$6, y$\geq$0.\\
\item Maximise Z=x+y, subject to x-y$\leq$-1,-x+y$\leq$0, x,y$\geq$0.\\
\item Reshma wishes to mix two types of food P and Q in such a way that the vitamin
contents of the mixture contain at least 8 units of vitamin A and 11 units of
vitamin B. Food P costs Rs 60/kg and Food Q costs Rs 80/kg. Food P contains
3 units/kg of Vitamin A and 5 units/kg of Vitamin B while food Q contains
4 units/kg of Vitamin A and 2 units/kg of vitamin B. Determine the minimum cost
of the mixture.\\
\item One kind of cake requires 200g of flour and 25g of fat, and another kind of cake
requires 100g of flour and 50g of fat. Find the maximum number of cakes which
can be made from 5kg of flour and 1 kg of fat assuming that there is no shortage
of the other ingredients used in making the cakes.\\
\item A factory makes tennis rackets and cricket bats. A tennis racket takes 1.5 hours
of machine time and 3 hours of craftman’s time in its making while a cricket bat
takes 3 hour of machine time and 1 hour of craftman’s time. In a day, the factory
has the availability of not more than 42 hours of machine time and 24 hours of
craftsman’s time.\\
(i) What number of rackets and bats must be made if the factory is to work
at full capacity?\\
(ii)If the profit on a racket and on a bat is Rs 20 and Rs 10 respectively, find
the maximum profit of the factory when it works at full capacity.\\
\item A manufacturer produces nuts and bolts. It takes 1 hour of work on machine A
and 3 hours on machine B to produce a package of nuts. It takes 3 hours on
machine A and 1 hour on machine B to produce a package of bolts. He earns a
profit of Rs17.50 per package on nuts and Rs 7.00 per package on bolts. How
many packages of each should be produced each day so as to maximise his
profit, if he operates his machines for at the most 12 hours a day?\\
\item A factory manufactures two types of screws, A and B. Each type of screw
requires the use of two machines, an automatic and a hand operated. It takes
4 minutes on the automatic and 6 minutes on hand operated machines to
manufacture a package of screws A, while it takes 6 minutes on automatic and
3 minutes on the hand operated machines to manufacture a package of screws
B. Each machine is available for at the most 4 hours on any day. The manufacturer
can sell a package of screws A at a profit of Rs 7 and screws B at a profit of
Rs 10. Assuming that he can sell all the screws he manufactures, how many
packages of each type should the factory owner produce in a day in order to
maximise his profit? Determine the maximum profit.\\
\item A cottage industry manufactures pedestal lamps and wooden shades, each
requiring the use of a grinding/cutting machine and a sprayer. It takes 2 hours on
grinding/cutting machine and 3 hours on the sprayer to manufacture a pedestal
lamp. It takes 1 hour on the grinding/cutting machine and 2 hours on the sprayer
to manufacture a shade. On any day, the sprayer is available for at the most 20
hours and the grinding/cutting machine for at the most 12 hours. The profit from
the sale of a lamp is Rs 5 and that from a shade is Rs 3. Assuming that the
manufacturer can sell all the lamps and shades that he produces, how should he
schedule his daily production in order to maximise his profit?\\
\item A company manufactures two types of novelty souvenirs made of plywood.
Souvenirs of type A require 5 minutes each for cutting and 10 minutes each for
assembling. Souvenirs of type B require 8 minutes each for cutting and 8 minutes
each for assembling. There are 3 hours 20 minutes available for cutting and 4
hours for assembling. The profit is Rs 5 each for type A and Rs 6 each for type
B souvenirs. How many souvenirs of each type should the company manufacture
in order to maximise the profit?\\
\item A merchant plans to sell two types of personal computers – a desktop model and
a portable model that will cost Rs 25000 and Rs 40000 respectively. He estimates
that the total monthly demand of computers will not exceed 250 units. Determine
the number of units of each type of computers which the merchant should stock
to get maximum profit if he does not want to invest more than Rs 70 lakhs and if
his profit on the desktop model is Rs 4500 and on portable model is Rs 5000.\\
\item A diet is to contain at least 80 units of vitamin A and 100 units of minerals. Two
foods$ F_{1}$ and $F_{2}$ are available. Food $F_{1}$ costs Rs 4 per unit food and $F_{2}$ costs
Rs 6 per unit. One unit of food $F_{1}$ contains 3 units of vitamin A and 4 units of
minerals. One unit of food $F_{2}$ contains 6 units of vitamin A and 3 units of minerals.
Formulate this as a linear programming problem. Find the minimum cost for diet
that consists of mixture of these two foods and also meets the minimal nutritional
requirements.\\
\item There are two types of fertilisers $F_{1}$ and $F_{2}$.$F_{1}$ consists of $10\%$ nitrogen and $6\%$
phosphoric acid and $F_{2}$ consists of $5\%$ nitrogen and $10\%$ phosphoric acid. After
testing the soil conditions, a farmer finds that she needs atleast 14 kg of nitrogen
and 14 kg of phosphoric acid for her crop. If $F_{1}$ costs Rs 6/kg and $F_{2}$ costs
Rs 5/kg, determine how much of each type of fertiliser should be used so that
nutrient requirements are met at a minimum cost. What is the minimum cost?\\
\item The corner points of the feasible region determined by the following system of
linear inequalities:
2x+y$\leq$10, x+3y$\leq$15, x,y$\geq$0 are (0,0), (5,0),(3,4) and (0,5).Let
Z=px+qy, where p,q$>$0.Condition on p and q so that the maximum of Z
occurs at both (3,4) and (0,5) is\\
(A) p = q\\
(B) p = 2q\\
(C) p = 3q\\
(D) q = 3p\\
\item Refer to Example 9. How many packets of each food should be used to maximise
the amount of vitamin A in the diet? What is the maximum amount of vitamin A
in the diet?\\
\item A farmer mixes two brands P and Q of cattle feed. Brand P, costing Rs 250 per
bag, contains 3 units of nutritional element A, 2.5 units of element B and 2 units
of element C. Brand Q costing Rs 200 per bag contains 1.5 units of nutritional
element A, 11.25 units of element B, and 3 units of element C. The minimum
requirements of nutrients A, B and C are 18 units, 45 units and 24 units respectively.
Determine the number of bags of each brand which should be mixed in order to
produce a mixture having a minimum cost per bag? What is the minimum cost of
the mixture per bag?\\
\item A dietician wishes to mix together two kinds of food X and Y in such a way that
the mixture contains at least 10 units of vitamin A, 12 units of vitamin B and
8 units of vitamin C. The vitamin contents of one kg food is given below:\\
\begin{tabular}{|c|c|c|c|}
\hline
\textbf{Food} &\textbf{Vitamin A} &\textbf{Vitamin B} & \textbf{VitaminC}\\
\hline
X & 1 & 2 & 3\\
\hline
Y &2 &2 &1\\
\hline


\end{tabular}\\
One kg of food X costs Rs 16 and one kg of food Y costs Rs 20. Find the least
cost of the mixture which will produce the required diet?\\
\item A manufacturer makes two types of toys A and B. Three machines are needed
for this purpose and the time (in minutes) required for each toy on the machines
is given below:\\
\begin{tabular}{|c|c|c|c|}
\hline
 \multicolumn{3}{|r}{\textbf{ Machines}}& \\ \cline{2-4}
\hline
\textbf {Types of toys}&\textbf{I}&\textbf{II}&\textbf{III}\\
\hline
A&12&18&6\\
\hline
 B&6&0&9\\
 \hline 

\end{tabular}



Each machine is available for a maximum of 6 hours per day. If the profit on
each toy of type A is Rs 7.50 and that on each toy of type B is Rs 5, show that 15
toys of type A and 30 of type B should be manufactured in a day to get maximum
profit.\\
\item An aeroplane can carry a maximum of 200 passengers. A profit of Rs 1000 is
made on each executive class ticket and a profit of Rs 600 is made on each
economy class ticket. The airline reserves at least 20 seats for executive class.
However, at least 4 times as many passengers prefer to travel by economy class
than by the executive class. Determine how many tickets of each type must be
sold in order to maximise the profit for the airline. What is the maximum profit?\\
\item Two godowns A and B have grain capacity of 100 quintals and 50 quintals
respectively. They supply to 3 ration shops, D, E and F whose requirements are
60, 50 and 40 quintals respectively. The cost of transportation per quintal from
the godowns to the shops are given in the following table:\\
\begin{tabular}{|c|c|c|}
\hline
 \multicolumn{2}{|l}{\textbf{ Transportation cost per qunital (in Rs)}}& \\ \cline{2-3}
\hline
\textbf {From/To}&\textbf{A}&\textbf{B}\\
\hline
D&6&4\\
\hline
 E&3&2\\
 \hline 
 F&2.50&3\\
 \hline

\end{tabular}\\


How should the supplies be transported in order that the transportation cost is
minimum? What is the minimum cost?\\
\item An oil company has two depots A and B with capacities of 7000 L and 4000 L
respectively. The company is to supply oil to three petrol pumps, D, E and F
whose requirements are 4500L, 3000L and 3500L respectively. The distances
(in km) between the depots and the petrol pumps is given in the following table:\\
\begin{tabular}{|c|c|c|}
\hline
 \multicolumn{2}{|l}{\textbf{Distance in (km.)}}& \\ \cline{2-3}
\hline
\textbf {From/To}&\textbf{A}&\textbf{B}\\
\hline
D&7&3\\
\hline
 E&6&4\\
 \hline 
 F&3&2\\
 \hline

\end{tabular}\\

Assuming that the transportation cost of 10 litres of oil is Re 1 per km, how
should the delivery be scheduled in order that the transportation cost is minimum?
What is the minimum cost?\\
\item A fruit grower can use two types of fertilizer in his garden, brand P and brand Q.
The amounts (in kg) of nitrogen, phosphoric acid, potash, and chlorine in a bag of
each brand are given in the table. Tests indicate that the garden needs at least
240 kg of phosphoric acid, at least 270 kg of potash and at most 310 kg of
chlorine.
If the grower wants to minimise the amount of nitrogen added to the garden,
how many bags of each brand should be used? What is the minimum amount of
nitrogen added in the garden?\\
\begin{tabular}{|c|c|c|}
\hline
 \multicolumn{2}{|r}{\textbf{ kg per bag}}& \\ \cline{1-3}
\hline
&\textbf{Brand P}&\textbf{Brand Q}\\
\hline
Nitrogen&3&3.5\\
\hline
Phospheric acid&1&2\\
\hline
Potash&3&1.5\\
\hline
Chlorine&1.5&2\\
\hline

\end{tabular}

\item Refer to Question 29. If the grower wants to maximise the amount of nitrogen
added to the garden, how many bags of each brand should be added? What is
the maximum amount of nitrogen added?\\
\item A toy company manufactures two types of dolls, A and B. Market research and
available resources have indicated that the combined production level should not
exceed 1200 dolls per week and the demand for dolls of type B is at most half of that
for dolls of type A. Further, the production level of dolls of type A can exceed three
times the production of dolls of other type by at most 600 units. If the company
makes profit of Rs 12 and Rs 16 per doll respectively on dolls A and B, how many of
each should be produced weekly in order to maximise the profit?





\end{enumerate}
%\end{document}
