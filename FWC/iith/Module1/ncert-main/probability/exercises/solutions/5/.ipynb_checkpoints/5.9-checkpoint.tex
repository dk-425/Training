Total number of cards =52 with 4 aces,48 non-ace's and we need to select 2 cards
so X can be 0 ,1 or 2\\ 

Let $A \in \{0,1\}$ represent the random variable, where 0 represents first card being an non ace, 1 represents first card being ace. \\
Let $B \in \{0,1\}$ represent the random variable, where 0 represents second card being an non-ace, 1 represents second card being ace 
\begin{table}[ht]
\caption{Probability for random variables}
\centering
\resizebox{\columnwidth}{!}{
\begin{tabular}{|c|c|c|c|}
\hline
{\pr{A=0}}& 48/52 &{\pr{A=1}}& 4/52  \\
\hline
\pr{B=0|A=0}&  47/51 &\pr{B=0|A=1}& 48/51 \\
\hline
{\pr{B=1|A=0}}& 4/51 &\pr{B=1|A=1}& 3/51  \\ 
\hline 
\end{tabular}}
\label{5.9:Tab:Tcr}
\end{table}\\
if A=1 then 3 aces left and if A=0 then\\ 4 aces left in remaining 51 cards\\ \\ 
Case 1: \emph{X} = 0
\begin{align}
\nonumber
&\implies \pr{X=0}=\pr{A=0,B=0}\\ \nonumber
&\quad\,\, =\pr{A=0}\times \pr{B=0|A=0}\\ \nonumber
&\implies\pr{X=0} =188/221\\
\end{align}

Case 2: \emph{X} = 1
\begin{align}
\nonumber
&\pr{X=1}=\pr{A=1,B=0}+\pr{A=0,B=1}\\ \nonumber 
&\pr{A=1,B=0}=\pr{A=1}\times \pr{B=0|A=1}\\ \nonumber
&\pr{A=1,B=0}=16/221\\ \nonumber
&\pr{A=0,B=1}=\pr{A=0}\times \pr{B=1|A=0}\\ \nonumber
&\pr{A=0,B=1} =16/221\\ \nonumber
&\implies\pr{X=1}\,=\frac{32}{221}\\
\end{align}
Case 3: \emph{X} = 2
\begin{align}
\nonumber
&\implies \pr{X=2}=\pr{A=1,B=1}\\ \nonumber
&\quad\,\, =\pr{A=1}\times\pr{B=1|A=1}\\ \nonumber
&\implies \pr{X=2}=1/221\\
\end{align}

 Now we know that E(X) denotes the average or expectation value which means that E(X) is the weighted average of all values X can take,each value being weighted by the probability of that particular event/value of X occurring\\  
 i.e E(X) is given by
 \begin{align}
      E(X) = {\sum_{i=0}^2 x_i\times \pr{x_i} }
 \end{align}

\begin{table}[ht]
    \caption{Probability for various \emph{X}}
    \centering
    \begin{tabular}{|c|c|c|c|}
        \hline
{\emph{X}} & 0 & 1 & 2  \\
\hline
{\pr{X}} &  188/221 &  32/221 &  1/221 \\
\hline
{\emph{X}$\times$ \pr{X}} & 0 & 32/221 & 2/221  \\
\hline 
\end{tabular}
\label{5.9:Tcr}
\end{table}
\begin{align}
\implies E(X) = \frac{32 +2}{221} =\frac{2}{13}   
\end{align}
Final answer E(x) = 2/13 or option 4