Let us consider orthogonal vectors $\vec{m_1}$ and $\vec{m_2}$ to the given normal vector $\vec{n}$. Let, $\vec{m}$ = $\myvec{a\\b\\c}$, then
\begin{align}
\vec{m^T}\vec{n} &= 0\\
\implies\myvec{a&b&c}\myvec{2\\-1\\2} &= 0\\
\implies2a-b+2c &= 0
\end{align}
Let a=1 and b=0 we get,
\begin{align}
\vec{m_1} &= \myvec{1\\0\\-1} \label{eq:solutions/3/7/eq:eq1}
\end{align}
Let a=0 and b=1 we get,
\begin{align}
\vec{m_2} &= \myvec{0\\1\\\frac{1}{2}} \label{eq:solutions/3/7/eq:eq2}
\end{align}
Let us solve the equation,
\begin{align}
\vec{M}\vec{x} &= \vec{b}\label{eq:solutions/3/7/eq:eq3}
\end{align}
Substituting \eqref{eq:solutions/3/7/eq:eq1} and \eqref{eq:solutions/3/7/eq:eq2} in \eqref{eq:solutions/3/7/eq:eq3},
\begin{align}
    \myvec{1&0\\0&1\\-1&\frac{1}{2}}\vec{x} &= \myvec{3\\-2\\1}\label{eq:solutions/3/7/eq:eq4}
\end{align}
To solve \eqref{eq:solutions/3/7/eq:eq4}, we will perform Singular Value Decomposition on $\vec{M}$ as follows,
\begin{align}
\vec{M}=\vec{U}\vec{S}\vec{V}^T\label{eq:solutions/3/7/eq:eq5}
\end{align}
Where the columns of $\vec{V}$ are the eigen vectors of $\vec{M}^T\vec{M}$ ,the columns of $\vec{U}$ are the eigen vectors of $\vec{M}\vec{M}^T$ and $\vec{S}$ is diagonal matrix of singular value of eigenvalues of $\vec{M}^T\vec{M}$.
\begin{align}
\vec{M}^T\vec{M}=\myvec{2&\frac{-1}{2}\\\frac{-1}{2}&\frac{5}{4}}\label{eq:solutions/3/7/eq:eq6}\\
\vec{M}\vec{M}^T=\myvec{1&0&-1\\0&1&\frac{1}{2}\\-1&\frac{1}{2}&\frac{5}{4}}
\end{align}
Substituting \eqref{eq:solutions/3/7/eq:eq5} in \eqref{eq:solutions/3/7/eq:eq3},
\begin{align}
\vec{U}\vec{S}\vec{V}^T\vec{x} & = \vec{b}\\
\implies\vec{x} &= \vec{V}\vec{S_+}\vec{U^T}\vec{b}\label{eq:solutions/3/7/eq:eq7}
\end{align}
Where $\vec{S_+}$ is Moore-Penrose Pseudo-Inverse of $\vec{S}$. \\
Let us calculate eigen values of $\vec{M}\vec{M}^T$,
\begin{align}
\mydet{\vec{M}\vec{M}^T - \lambda\vec{I}} &= 0\\
\implies\myvec{1-\lambda&0&-1\\0&1-\lambda&\frac{1}{2}\\-1&\frac{1}{2}&\frac{5}{4}-\lambda} &=0\\
\implies\lambda^3-\frac{13}{4}\lambda^2+\frac{9}{4}\lambda &=0 \label{eq:solutions/3/7/eq:eq8}
\end{align}
From equation \eqref{eq:solutions/3/7/eq:eq8} eigen values of $\vec{M}\vec{M}^T$ are,
\begin{align}
\lambda_1 = \frac{9}{4} \quad
\lambda_2 = 1 \quad
\lambda_3 = 0
\end{align}
The eigen vectors of $\vec{M}\vec{M}^T$ are,
\begin{align}
\vec{u}_1=\myvec{-\frac{4}{5}\\\frac{2}{5}\\1}\quad
\vec{u}_2=\myvec{\frac{1}{2}\\1\\0}\quad
\vec{u}_3=\myvec{1\\-\frac{1}{2}\\1}\label{eq:solutions/3/7/eq:eq9}
\end{align}
Normalizing the eigen vectors in equation \eqref{eq:solutions/3/7/eq:eq9}
\begin{align}
\vec{u}_1=\myvec{-\frac{4}{3\sqrt{5}}\\\frac{2}{3\sqrt{5}}\\\frac{\sqrt{5}}{3}}\quad
\vec{u}_2=\myvec{\frac{1}{\sqrt{5}}\\\frac{2}{\sqrt{5}}\\0}\quad
\vec{u}_3=\myvec{\frac{2}{3}\\-\frac{1}{3}\\\frac{2}{3}}
\end{align}
Hence we obtain $\vec{U}$ as follows,
\begin{align}
\vec{U}=\myvec{-\frac{4}{3\sqrt{5}}&\frac{1}{\sqrt{5}}&\frac{2}{3}\\\frac{2}{3\sqrt{5}}&\frac{2}{\sqrt{5}}&-\frac{1}{3}\\\frac{\sqrt{5}}{3}&0&\frac{2}{3}}\label{eq:solutions/3/7/eq:eq10}
\end{align}
After computing the singular values from eigen values $\lambda_1, \lambda_2, \lambda_3$ we get $\vec{S}$ as follows,
\begin{align}
\vec{S}=\myvec{\frac{9}{4}&0\\0&1\\0&0}
\end{align}
Now, lets calculate eigen values of $\vec{M}^T\vec{M}$,
\begin{align}
\mydet{\vec{M}^T\vec{M} - \lambda\vec{I}} &= 0\\
\implies\myvec{2-\lambda&-\frac{1}{2}\\-\frac{1}{2}&\frac{5}{4}-\lambda} &=0\\
\implies\lambda^2-\frac{13}{4}\lambda+\frac{9}{4} &=0
\end{align}
Hence eigen values of $\vec{M}^T\vec{M}$ are,
\begin{align}
\lambda_1 = \frac{9}{4}\quad
\lambda_2 = 1
\end{align}
Hence the eigen vectors of $\vec{M}^T\vec{M}$ are,
\begin{align}
\vec{v}_1=\myvec{-2\\1} \quad
\vec{v}_2=\myvec{\frac{1}{2}\\1}
\end{align}
Normalizing the eigen vectors,
\begin{align}
\vec{v}_1=\myvec{-\frac{2}{\sqrt{5}}\\\frac{1}{\sqrt{5}}} \quad
\vec{v}_2=\myvec{\frac{1}{\sqrt{5}}\\\frac{2}{\sqrt{5}}}
\end{align}
Hence we obtain $\vec{V}$ as,
\begin{align}
\vec{V}=\myvec{-\frac{2}{\sqrt{5}}&\frac{1}{\sqrt{5}}\\\frac{1}{\sqrt{5}}&\frac{2}{\sqrt{5}}}
\end{align}
From \eqref{eq:solutions/3/7/eq:eq3}, the Singular Value Decomposition of $\vec{M}$ is as follows,
\begin{align}
\vec{M} = \myvec{-\frac{4}{3\sqrt{5}}&\frac{1}{\sqrt{5}}&\frac{2}{3}\\\frac{2}{3\sqrt{5}}&\frac{2}{\sqrt{5}}&-\frac{1}{3}\\\frac{\sqrt{5}}{3}&0&\frac{2}{3}}\myvec{\frac{9}{4}&0\\0&1\\0&0}\myvec{-\frac{2}{\sqrt{5}}&\frac{1}{\sqrt{5}}\\\frac{1}{\sqrt{5}}&\frac{2}{\sqrt{5}}}^T
\end{align}
Now, Moore-Penrose Pseudo inverse of $\vec{S}$ is given by,
\begin{align}
\vec{S_+} = \myvec{\frac{2}{3}&0&0\\0&1&0}
\end{align}
From \eqref{eq:solutions/3/7/eq:eq7} we get,
\begin{align}
\vec{U}^T\vec{b}&=\myvec{-\frac{11}{3\sqrt{5}}\\-\frac{1}{\sqrt{5}}\\\frac{10}{3}}\\
\vec{S_+}\vec{U}^T\vec{b}&=\myvec{-\frac{22}{9\sqrt{5}}\\-\frac{1}{\sqrt{5}}}\\
\vec{x} = \vec{V}\vec{S_+}\vec{U}^T\vec{b} &= \myvec{\frac{7}{9}\\-\frac{8}{9}} \label{eq:solutions/3/7/eq:eq11}
\end{align}
Verifying the solution of \eqref{eq:solutions/3/7/eq:eq11} using,
\begin{align}
\vec{M}^T\vec{M}\vec{x} = \vec{M}^T\vec{b}\label{eq:solutions/3/7/eqVerify}
\end{align}
Evaluating the R.H.S in \eqref{eq:solutions/3/7/eqVerify} we get,
\begin{align}
\vec{M}^T\vec{M}\vec{x} &= \myvec{2\\-\frac{3}{2}}\\
\implies\myvec{2&-\frac{1}{2}\\-\frac{1}{2}&\frac{5}{4}}\vec{x} &= \myvec{2\\-\frac{3}{2}}\label{eq:solutions/3/7/eq:eq12}
\end{align}
Solving the augmented matrix of \eqref{eq:solutions/3/7/eq:eq12} we get,
\begin{align}
\myvec{2&-\frac{1}{2}&2\\-\frac{1}{2}&\frac{5}{4}&-\frac{3}{2}} &\xleftrightarrow{R_1=\frac{R_1}{2}}\myvec{1&-\frac{1}{4}&1\\-\frac{1}{2}&\frac{5}{4}&-\frac{3}{2}}\\
&\xleftrightarrow{R_2=R_2+\frac{R_1}{2}}\myvec{1&-\frac{1}{4}&1\\0&\frac{9}{8}&-1}\\
&\xleftrightarrow{R_2=\frac{8}{9}R_2}\myvec{1&-\frac{1}{4}&1\\0&1&-\frac{8}{9}}\\
&\xleftrightarrow{R_1=R_1+\frac{R_2}{4}}\myvec{1&0&\frac{7}{9}\\0&1&-\frac{8}{9}}\label{eq:solutions/3/7/eq:eq13}
\end{align}
From equation \eqref{eq:solutions/3/7/eq:eq13}, solution is given by,
\begin{align}
\vec{x}=\myvec{\frac{7}{9}\\-\frac{8}{9}}\label{eq:solutions/3/7/eq:eq14}
\end{align}
Comparing results of $\vec{x}$ from \eqref{eq:solutions/3/7/eq:eq11} and \eqref{eq:solutions/3/7/eq:eq14}, we can say that the solution is verified.
