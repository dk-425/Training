In general, any complex number can be expressed in matrix representation as follows:
\begin{align}  \label{eq:sol_complex_14gen_1}
& \myvec{a1\\ a2} = \myvec{a1 & -a2\\ a2 & a1}\myvec{1\\ 0}
\end{align}
Converting complex number to matrix form:
\begin{align}  
& \frac{\myvec{1\\ 2}}{\myvec{1\\ -3}} = \myvec{1 & -2\\ 2 &1}\myvec{1 & 3\\ -3 &1}^{-1}\myvec{1\\ 0}\label{eq:sol_complex_14equ_1}\\
&\myvec{1 & 3\\ -3 &1}^{-1} = \myvec{1/10 & -3/10\\ 3/10 &1/10}\label{eq:sol_complex_14eq_2}
\end{align}
Sub \eqref{eq:sol_complex_14eq_2} in \eqref{eq:sol_complex_14equ_1},
\begin{align}  
& \frac{\myvec{1\\ 2}}{\myvec{1\\ -3}} = \myvec{1 & -2\\ 2 &1}\myvec{1/10 & -3/10\\ 3/10 &1/10}\myvec{1\\ 0}\label{eq:sol_complex_14eq_3}\\
& = \myvec{1 & -2\\ 2 &1}\myvec{1/10\\ 3/10}\\
& = \myvec{-5/10\\ 5/10}\\
& \implies \boxed{\frac{\myvec{1\\ 2}}{\myvec{1\\ -3}} = \myvec{-1/2\\ 1/2}} \label{eq:sol_complex_14fin_eq}
\end{align}
From \eqref{eq:sol_complex_14fin_eq},
The modulus and argument of the complex number is,
\begin{align}
& r = \norm{\myvec{-1/2\\ 1/2}} = \frac{1}{\sqrt{2}}\label{eq:sol_complex_14mod}\\
& \tan\theta = -1 \implies \theta = 180^{\circ}-45^{\circ} = 135^{\circ} \label{eq:sol_complex_14arg}
\end{align}


