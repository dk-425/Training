\begin{align}
\myvec{1\\1} &= \sqrt{2}\myvec{\frac{1}{\sqrt{2}}\\ \frac{1}{\sqrt{2}}}
\\
&= \sqrt{2}\myvec{\cos 45\degree \\ \sin 45\degree}
\label{eq:3.4.7}
\end{align}
In the above, the modulus is $\norm{\myvec{1\\1}}=\sqrt{2}$ and the argument is $45 \degree$.
Similarly, 
\begin{align}
\myvec{1\\-1} &= \sqrt{2}\myvec{\cos 45\degree \\ -\sin 45\degree}
\\
\implies \myvec{1\\-1}^{-1} &= \frac{1}{\sqrt{2}}\myvec{\cos 45\degree \\ \sin 45\degree}
\end{align}
Using the matrix representation, 
\begin{align}
\frac{\myvec{1\\1}}{\myvec{1\\-1}} &= \myvec{\cos45\degree&-\sin45\degree\\\sin45\degree&\cos45\degree}
\nonumber \\
&\quad \times  \myvec{\cos45\degree&-\sin45\degree\\\sin45\degree&\cos45\degree}
\myvec{1\\0}
\\
 &= \myvec{\cos 90\degree\\ \sin 90\degree}= 1 \phase{90\degree}\end{align}
%
In general, if
\begin{align}
\vec{z}_1 &= r_1\myvec{\cos\theta_1\\\sin\theta_1}, \vec{z}_2 = r_2\myvec{\cos\theta_2\\\sin\theta_2},
\\
\vec{z}_1\vec{z}_2 &= r_1r_2\myvec{\cos(\theta_1+\theta_2)\\\sin(\theta_1+\theta_2)}.
\end{align}
Similarly, from \eqref{eq:3.4.7},
\begin{align}
\frac{1}{\myvec{1\\1}} &= \frac{1}{\sqrt{2}}\myvec{\cos 45\degree \\ -\sin 45\degree}
\\
&= \frac{1}{\sqrt{2}}\phase{-45\degree}
\end{align}
