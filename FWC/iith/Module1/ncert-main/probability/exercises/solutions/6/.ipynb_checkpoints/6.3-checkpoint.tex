Given 
\begin{align}
Pr(B | A)=1. 
\end{align}
By definition,
\begin{align}
Pr(B|A)=\frac{Pr(AB)}{Pr(A)}
\end{align}
\begin{align}
&\implies\frac{Pr(AB)}{Pr(A)} = 1\\
&\implies Pr(AB) = Pr(A)\label{1}\\
&\implies AB= A\label{2}
\end{align}
\begin{enumerate}[label={\Alph*)}]
\item Take any 
\begin{align}
    X \in A
\end{align} . From \eqref{2}, we get
\begin{align}
     X \in AB
\end{align}is also true.\\
Therefore, for any 
\begin{align}
  X \in A \\
  \implies X \in B   
\end{align}
    \begin{align}
     A \subseteq B 
\end{align}is also true.\\
    But, since A and B are two events,
    \begin{align}
       A\neq B 
    \end{align}. Hence,
    \begin{align}
    A \subset B
\end{align}
Therefore, option (A) is correct.
% \begin{figure}[h]
%     \centering
%     \includegraphics[width=8cm]{venn.png}
    
%     \label*{Venn diagram}
% \end{figure}
\item If 
\begin{align}
  B \subset A  
\end{align}
Then, 
\begin{align}
AB=B.
\end{align}
\begin{align}
\implies Pr(AB)=Pr(B)
\end{align}
But, from \eqref{1}, we have, 
\begin{align}
&Pr(AB)=Pr(A)\\
\implies &Pr(AB)=Pr(A)=Pr(B)
\end{align}
But,since A and B are two events, 
\begin{align}
    A\neq B
\end{align}. Hence, option (B) is incorrect.
 \item If 
 \begin{align}
     B=\phi
 \end{align}
\begin{align}
\implies Pr(AB)=0
\end{align}From \eqref{1}, we know that,
\begin{align}
& Pr(AB)=Pr(A)\\
\implies & Pr(AB)=Pr(A)=0
\end{align}
But,from the given data, we know that 
\begin{align}
     Pr(A) \neq 0
 \end{align}
Therefore,option C is incorrect.
\item If 
\begin{align}
     A=\phi
 \end{align}
\begin{align}
\implies Pr(A)=0
\end{align}
But,from the given data, we know that 
\begin{align}
     Pr(A) \neq 0
 \end{align}
Therefore,option D is incorrect.
\end{enumerate}