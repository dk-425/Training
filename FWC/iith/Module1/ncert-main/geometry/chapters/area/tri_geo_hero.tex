\renewcommand{\theequation}{\theenumi}
\begin{enumerate}[label=\arabic*.,ref=\thesubsection.\theenumi]
\numberwithin{equation}{enumi}

\item Find Hero's formula for the area of a triangle.
\\
\solution 
%In Fig. \ref{fig:rt_triangle}, from Baudhayana's theorem, 
%\begin{align}
%\label{eq:tri_geo_baudh}
%b^2 = a^2+c^2 &
%\\
%=b^2\cos^2C+b^2\sin^2C &
%\\
%\implies \cos^2C+\sin^2C &= 1
%\end{align}
%
%In Fig. \ref{fig:tri_const_ex_cos_form}, 
From \eqref{prob:tri_area_sin}, the area of $\triangle ABC$ is 
{\footnotesize
\begin{align}
\label{eq:tri_geo_area_sin_form}
 \frac{1}{2}ab\sin C
%\\
&=\frac{1}{2}ab\sqrt{1-\cos^2C} 
\quad \brak{\text{from } \eqref{eq:tri_sin_cos_id}
%\eqref{eq:tri_geo_baudh}
}
\\
&=\frac{1}{2}ab\sqrt{1-\brak{\frac{a^2+b^2-c^2}{2ab}}^2} \brak{\text{from } \eqref{eq:tri_cos_form}
}
\\
&=\frac{1}{4}\sqrt{\brak{2ab}^2-\brak{a^2+b^2-c^2}}
\\
&=\frac{1}{4}\sqrt{\brak{2ab+a^2+b^2-c^2}\brak{2ab-a^2-b^2+c^2}}
\\
&= \frac{1}{4}\sqrt{\cbrak{\brak{a+b}^2-c^2}\cbrak{c^2-\brak{a-b}^2}}
\\
&= \frac{1}{4}\sqrt{\brak{a+b+c}\brak{a+b-c}\brak{a+c-b}\brak{b+c-a}}
\label{eq:tri_ex_hero_temp}
\end{align}
}
Substituting 
%
\begin{align}
s=\frac{a+b+c}{2}
\end{align}
%
in \eqref{eq:tri_ex_hero_temp}, the area of $\triangle ABC$ is 
%
\begin{align}
\label{eq:tri_area_hero}
\sqrt{s\brak{s-a}\brak{s-b}\brak{s-c}}
\end{align}
%
This is known as Hero's formula.
\item Find the area of $\triangle ABC$ in Fig. \ref{fig:tri_rect}.
\\
\solution The desired are is computed using \eqref{eq:tri_area_hero} by the following 
the python code.
\begin{lstlisting}
codes/triangle/tri_area_hero.py
\end{lstlisting}
%

\item Show that the sum of two sides of a triangle is always greater than the third side.
\\
\solution In \eqref{eq:tri_area_hero}, all terms under the square roots should be positive.  Hence,
%
\begin{align}
\label{eq:tri_hero_ineq}
\begin{split}
s-a &>0
\\
s-b &>0
\\
s-c &>0
\end{split}
\end{align}
resulting in 
%
\begin{align}
\begin{split}
\label{eq:tri_sum_ineq}
b+c &>a
\\
c+a &>b
\\
a+b &>c
\end{split}
\end{align}
\end{enumerate}


